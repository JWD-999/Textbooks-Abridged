\documentclass[../main]{subfiles}

\begin{document}
 
 \section{Subgoups}
 
 \subsection{Definition and Examples}
 
 
 \begin{dfn}
  Let $G$ be a group. The subset $H$ of $G$ is a \textit{subgroup} of $G$ if $H$ is nonempty and $H$ is closed under products and inverse (i.e, $x,y\in H$ implies $x\in H$ and $xy\in H$). If $H$ is a subgroup of $G$ we shall write $H\leq G$.
 \end{dfn}

 
 \begin{prop}
  (The Subgroup Criterion) A subset $H$ of a group $G$ is a subgroup if and only if 
  \begin{enumerate}
   \item $H\neq \emptyset$, and 
   \item for all $x,y \in H, xy^{-1}\in H$
  \end{enumerate}
 \end{prop}
 
 
 \subsection{Centralizers and Nomalizers, Stabilizers and Kernels}
 
 
 Let $G$ be a group and $A$ a nonempty subset of $G$.
 
 
 \begin{dfn}
  The \textit{centralizer} of $A$ in $G$ is $C_G(A)=\{g\in G \mid gag^{-1}=a $ for all $a\in A\}$. Note that this is the set of elements of $G$ which commute with every element of $A$. Note that $C_g(A) \leq G$.
 \end{dfn}
 
 
 \begin{dfn}
  The \textit{center} of $G$ is the set $Z(G)=\{g\in G\mid gx=xg$ for all $x\in G\}$. Note that, $Z(G)=C_G(G)$, thus $Z(G)\leq G$.
 \end{dfn}


 \begin{dfn}
  Define $gAg^{-1}=\{gag^{-1} \mid a\in A\}$. The \textit{normalizer} of $A$ in $G$ is the set $N_G(A)=\{g\in G \mid gAg^{-1}=A\}$. Note that, $C_G(A)\leq N_G(A)\leq G$.
 \end{dfn}

 
 \subsection{Cyclic Groups and Cyclic Subgroups}
 
 
 \begin{dfn}
  A group $H$ is \textit{cyclic} if $H$ can be generated by a single element, i.e, there exist some $x\in H$ such that $H= \{x^n\mid n\in \Z\}$ when using multiplicative notation and $H= \{nx \mid n\in \Z\}$ when using additive notation. In either case we write $H = \langle x \rangle$. 
 \end{dfn}
 
 
 \begin{prop}
  If $H=\langle x\rangle$, then $|H|=|x|$. Moreover,
  \begin{enumerate}
   \item if $|H| = n < \infty $, then $x^n = 1$ and $1,x,x^2,\ldots,x^{n-1}$ are all distinct elements of H, and 
   \item if $|H|= \infty$, then $x^n \neq 1$ for all $n \neq 0$ and $x^a \neq x^b$ for all $a\neq b \in \Z$.
  \end{enumerate}
 \end{prop}
 
 
 \begin{prop}
  Let $G$ be an arbitrary group, $x\in G$ and let $m,n \in \Z$. If $x^n = 1$ and $x^m = 1$ then $x^d = 1$ where $d=(m,n)$. In particular, if $x^m=1$ for some $m\in \Z$ then $|x|$ divides $m$.
 \end{prop}
 
 
 \begin{thm}
  Any two cyclic groups of the same order are isomorphic. Moreover, 
  \begin{enumerate}
   \item if $n\in \Z^+$ and $\langle x \rangle$ and $\langle y \rangle$ are both cyclic groups of orger n, then the map
   \begin{align*}
    \phi \colon \langle x \rangle \to \langle y \rangle \\
    x^k \mapsto y^k
   \end{align*}
   is well defined and is an isomorphism
   \item if $\langle x \rangle$ is an infinite cyclic group, the map 
   \begin{align*}
    \phi \colon \Z \to \langle x \rangle \\
    k \mapsto x^k
   \end{align*}
   is well defined and is an isomorphism
  \end{enumerate}
 \end{thm}

 
 \begin{prop}
  Let $G$ be a group, let $x\in G$ and let $a\in \Z-\{0\}.$
  \begin{enumerate}
   \item If $|x| = \infty$, then $|x^a| = \infty$.
   \item If $|x| = n < \infty$, then $|x^a|=\frac{n}{(n,a)}$.
   \item In particular, if $|x|=n<\infty$ and $a$ is a postive integer dividing $n$, then $|x^a|=\frac{n}{a}.$
  \end{enumerate}
 \end{prop}

 
 \begin{prop}
  Let $H=\langle x\rangle$.
  \begin{enumerate}
   \item Assume $|x|= \infty$. Then $H=\langle x^a \rangle$ if and only if $a=\pm 1$.
   \item Assume $|x| = n <\infty$. Then $H=\langle x^a \rangle$ if and only if $(a,n)=1$. In particular, the number of generators of $H$ is $\phi (n)$ (where $\phi$ is Euler's $\phi$-function)
  \end{enumerate}
 \end{prop}

 
 
 \begin{thm}
  Let $H=\langle x \rangle$ be a cyclic group. 
  \begin{enumerate}
   \item Every subgroup of $H$ is cyclic. More precisely, if $K\leq H$, then either $K=\{1\}$ or $K=\langle x^d \rangle$, where $d$ is the smallest positive integer such that $x^d \in K$.
   \item If $|H| = \infty$, then for any distinct nonnegative integers $a$ and $b$, $\langle x^a \rangle \neq \langle x^b \rangle$. Furthermore, for every integer $m$, $\langle x^m \rangle = \langle x^{|m|} \rangle$, where $|m|$ denotes the absolute value of m, so that the nontrival sungroups of $H$ correspond bijectively with the integers $1,2,3,\ldots$.
   \item If $|H| = n <\infty$, then for each positive integer $a$ dividing $n$ there is a unique subgroup of $H$ of order $a$. This subgroup is the cyclic group $\langle x^d \rangle$, where $d=\frac{n}{a}$. Furthermore, for every integer $m$, $\langle x^m \rangle = \langle x^{(n,m)} \rangle$, so that the subgroups of $H$ correspond bijectively with the positive divisors of n.
  \end{enumerate}
 \end{thm}

 
 \subsection{Subgroups Generated by Subsets of a Group}
 
 
 \begin{prop}
  If $\mathcal{A}$ is any nonempty collection of subgroups of $G$, then the intersection of all members of $\mathcal{A}$ is also a subgroup of $G$.
 \end{prop}

 
 \begin{dfn}
  If $A$ is any subset of the group $G$ define
   \[\langle A \rangle = \bigcap\limits_{\substack{A \subseteq H \\ H\leq G}} H.\]
  This is called the \textit{subgroup of $G$ generated by $A$}.
 \end{dfn}
 
 
 \begin{nt}
  $\langle A \rangle = \{a_1^{\epsilon_1} a_2^{\epsilon_2}\ldots a_n^{\epsilon_n} \mid n\in \Z, n\geq 0$ and $a_i\in A, \epsilon_i = \pm 1$ for each $i\}$.
 \end{nt}


\end{document}
