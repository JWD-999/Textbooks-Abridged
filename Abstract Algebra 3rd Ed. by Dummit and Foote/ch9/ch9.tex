\documentclass[../main]{subfiles}

\begin{document}
 
\section{Polynomial Rings}

In this chapter the ring $R$ will always be a commutative ring with identity $1 \neq 0$.

\subsection{Definitions and Basic Properties}


\begin{prop}
 Let $R$ be an integral domain. Then 
 \begin{enumerate}
  \item degree $p(x) q(x) =$ degree $p(x)$ $+$ degree $q(x)$ if $p(x), q(x)$ are nonzero
  
  \item the units of $R[x]$ are just the units of $R$
  
  \item $R[x]$ is an integral domain.
 \end{enumerate}
\end{prop}


\begin{prop}
 Let $I$ be an ideal of the ring $R$ and let $(I) = I[x]$ denote the ideal of $R[x]$ generated by $I$ (the set of polynomials with coefficients in $I$). Then
 \[ R[x]/(I) \cong (R/I)[x]. \]
 In particular, if $I$ is a prime ideal of $R$ then $(I)$ is a prime ideal of $R[x]$
\end{prop}


\begin{dfn}
 The \textit{polynomial ring in variables $x_1, x_2, \ldots, x_n$ with coefficients in $R$}, denoted $R[x_1, x_2, \dots, x_n]$ is defined inductively by 
 \[ R[x_1, x_2, \dots, x_n] = R[x_1, x_2, \dots, x_{n-1}][x_n]. \]
\end{dfn}


\subsection{Polynomial Rings over Fields I}


\begin{thm}
 Let $F$ be a field. The polynomial ring $F[x]$ is a Euclidean Domain. Specifically, if $a(x)$ and $b(x)$ are two polynomials in $F[x]$ with $b(x)$ nonzero, then there are unique $q(x)$ and $r(x)$ in $F[x]$ such that 
 \[ a(x) = q(x) b(x) + r(x) \qquad \text{with } r(x) = 0 \text{ or degree } r(x) < \text{ degree } b(x). \]
\end{thm}


\begin{cor}
 If $F$ is a field, then $F[x]$ is a Principal Ideal Domain and a Unique Factorization Domain.
\end{cor}


\subsection{Polynomial Rings that are Unique Factorization Domains}


\begin{prop}
 (Gauss' Lemma) Let $R$ be a Unique Factorization Domain with field of fractions $F$ and let $p(x) \in R[x]$. If $p(x)$ is reducible in $F[x]$ then $p(x)$ is reducible in $R[x]$. More precisely, if $p(a) = A(x) B(x)$ for some nonconstant polynomials $A(x),B(x) \in F[x]$, then there are nonzero elements $r,s \in F$ such that $r A(x) = a(x)$ and $s B(x) = b(x)$ both lie in $R[x]$ and $p(x) = a(x) b(x)$ is a factorization in $R[x]$.
\end{prop}


\begin{cor}
 Let $R$ be a Unique Factorization Domain, let $F$ be its field of fractions and let $p(x) \in R[x]$. Suppose the greatest common divisor of the coefficients of $p(x)$ is 1. Then $p(x)$ is irreducible in $R[x]$ if and only if it is irreducible in $F[x]$. In particular, if $p(x)$ is a monic polynomial that is irreducible in $R[x]$, then $p(x)$ is irreducible in $F[x]$.
\end{cor}


\begin{thm}
 $R$ is a Unique Factorization Domain if and only if $R[x]$ is a Unique Factorization Domain.
\end{thm}


\begin{cor}
 If $R$ is a Unique Factorization Domain, then a polynomial ring in an arbitrary number of variables with coefficients in $R$ is also a Unique Factorization Domain.
\end{cor}


\subsection{Irreducibility Criteria}


\begin{prop}
 Let $F$ be a field and let $p(x) \in F[x]$. Then $p(x)$ has a factor of degree one if and only if $p(x)$ has a root in $F$, i.e., there is an $\alpha \in F$ with $p(\alpha) = 0$.
\end{prop}


\begin{prop}
 A polynomial of degree two or three over a field $F$ is reducible if and only if it has a root in $F$.
\end{prop}


\begin{prop}
 Let $p(x) = a_n x^n + a_{n-1} x^{n-1} + \ldots + a_0$ be a polynomial of degree $n$ with integer coefficients. If $r/s \in \Q$ is in lowest terms (i.e., $r$ and $s$ are relatively prime integers) and $r/s$ is a root of $p(x)$, then $r$ divides the constant term and $s$ divides the leading coefficient of $p(x)$: $r|a_0$ and $s|a_n$. In particular, If $p(x)$ is a monic polynomial with integer coefficients and $p(d) \neq 0$ for all integers $d$ dividing the constant term of $p(x)$, then $p(x)$ has no roots in $\Q$.
\end{prop}


\begin{prop}
 Let $I$ be a proper ideal in the integral domain $R$ and let $p(x)$ be a nonconstant monic polynomial in $R[x]$. If the image of $p(x)$ in $(R/I)[x]$ cannot be factored in $(R/I)[x]$ into two polynomials of smaller degree, then $p(x)$ is irreducible in $R[x]$.
\end{prop}


\begin{prop}
 (Eisenstein's Criterion) Let $P$ be a prime ideal of the integral domain $R$ and let $f(x) = x^n + a_{n-1} x^{n-1} + \ldots + a_0$ be a polynomial in $R[x]$ (here $n\geq 1$). Suppose $a_{n-1}, \ldots, a_0$ are all elements of $P$ and suppose $a_0$ is not an element of $P^2$. Then $f(x)$ is irreducible in $R[x]$.
\end{prop}


\begin{cor}
 (Eisenstein's Criterion for $\Z[x]$) Let $p$ be a prime in $\Z$ and let $f(x) = x^n + a_{n-1} x^{n-1} + \ldots + a_0 \in \Z[x], n \geq 1$. Suppose $p$ divides $a_i$ for all $i \in \{ 0, 1, \ldots, n-1 \}$ but that $p^2$ does not divide $a_0$. Then $f(x)$ is irreducible in both $\Z[z]$ and $\Q[x]$.
\end{cor}


\subsection{Polynomial Rings over Fields II}

Let $F$ be a field.


\begin{prop}
 The maximal ideal of $F[x]$ are the ideals $(f(x))$ generated by irreducible polynomials $f(x)$. In particular, $F[x]/(f(x))$ is a field if and only if $f(x)$ is irreducible.
\end{prop}


\begin{prop}
 Let $g(x)$ be a nonconstant monic element of $F[x]$ and let 
 \[ g(x) = f_1(x)^{n_1} f_2(x)^{n_2} \cdots f_k(x)^{n_k} \]
 be its factorization into irreducibles, where the $f_i(x)$ are distinct. Then we have the following isomorphism of rings:
 \[ F[x]/(g(x)) \cong F[x]/(f_1(x)^{n_1}) \times F[x]/(f_2(x)^{n_2}) \times \cdots \times F[x]/(f_k(x)^{n_k}). \]
\end{prop}


\begin{prop}
 If the polynomial $f(x)$ has roots $\alpha_1, \alpha_2, \ldots \alpha_k$ in $F$ (not necessarily distinct), then $f(x)$ has $(x-\alpha_1) \cdots (x-\alpha_k)$ as a factor. In particular, a polynomial of degree $n$ in one variable over a field $F$ has at most $n$ roots in $F$, even counted with multiplicity.
\end{prop}


\begin{prop}
 A finite subgroup of the multiplicative group of a field is cyclic. In particular, if $F$ is a finite field, then the multiplicative group $F^\times$ of nonzero elements of $F$ is a cyclic group.
\end{prop}


\begin{cor}
 Let $p$ be a prime. The multiplicative group $(\Z/p\Z)^\times$ of nonzero residue classes mod $p$ is cyclic.
\end{cor}


\begin{cor}
 Let $n\geq 2$ be an integer with factorization $n = p_1^{\alpha_1} p_2^{\alpha_2} \cdots p_r^{\alpha_r}$ in $\Z$, where $p_1, \ldots, p_r$ are distinct primes. We have the following isomorphisms of (multiplicative) groups
 \begin{enumerate}
  \item $(\Z/n\Z)^\times \cong (\Z/p_1^{\alpha_1}\Z)^\times \times (\Z/p_2^{\alpha_2}\Z)^\times \times \cdots \times (\Z/p_r^{\alpha_r}\Z)^\times$
  
  \item $(\Z/2^\alpha \Z)^\times$ is the direct product of a cyclic group of order 2 and a cyclic group of order $2^{\alpha - 2}$, for all $\alpha \geq 2$
  
  \item $(\Z/p^\alpha \Z)^\times$ is a cyclic group of order $p^{\alpha - 1}(p-1)$, for all odd primes $p$.
 \end{enumerate}
\end{cor}


\end{document}
