\documentclass[../main]{subfiles}

\begin{document}
 
\section{Polynomial Rings}

In this chapter the ring $R$ will always be a commutative ring with identity $1 \neq 0$.

\subsection{Definitions and Basic Properties}


\begin{prop}
 Let $R$ be an integral domain. Then 
 \begin{enumerate}
  \item degree $p(x) q(x) =$ degree $p(x)$ $+$ degree $q(x)$ if $p(x), q(x)$ are nonzero
  
  \item the units of $R[x]$ are just the units of $R$
  
  \item $R[x]$ is an integral domain.
 \end{enumerate}
\end{prop}


\begin{prop}
 Let $I$ be an ideal of the ring $R$ and let $(I) = I[x]$ denote the ideal of $R[x]$ generated by $I$ (the set of polynomials with coefficients in $I$). Then
 \[ R[x]/(I) \cong (R/I)[x]. \]
 In particular, if $I$ is a prime ideal of $R$ then $(I)$ is a prime ideal of $R[x]$
\end{prop}


\begin{dfn}
 The \textit{polynomial ring in variables $x_1, x_2, \ldots, x_n$ with coefficients in $R$}, denoted $R[x_1, x_2, \dots, x_n]$ is defined inductively by 
 \[ R[x_1, x_2, \dots, x_n] = R[x_1, x_2, \dots, x_{n-1}][x_n]. \]
\end{dfn}


\subsection{Polynomial Rings over Fields I}


\begin{thm}
 Let $F$ be a field. The polynomial ring $F[x]$ is a Euclidean Domain. Specifically, if $a(x)$ and $b(x)$ are two polynomials in $F[x]$ with $b(x)$ nonzero, then there are unique $q(x)$ and $r(x)$ in $F[x]$ such that 
 \[ a(x) = q(x) b(x) + r(x) \qquad \text{with } r(x) = 0 \text{ or degree } r(x) < \text{ degree } b(x). \]
\end{thm}


\begin{cor}
 If $F$ is a field, then $F[x]$ is a Principal Ideal Domain and a Unique Factorization Domain.
\end{cor}


\subsection{Polynomial Rings that are Unique Factorization Domains}





 
 
 
 
 
 
 
 
 
 
 
 
 
 
 
 
 
 
 
\end{document}
