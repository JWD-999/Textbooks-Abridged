\documentclass[../main]{subfiles}


\begin{document}
 
 
 \section{Group Actions}
 
 
 \subsection{Group Actions and Permutation Representations}
 
 
 \begin{dfn} Let $G$ be a group acting on a set $A$
  \begin{enumerate}
    \item The \textit{kernel} of the action is the set of elements of $G$ that act trivially on every element of $A$: $\{g\in G \mid g \cdot a = a $ for all $a\in A\}$.
    \item For each $a\in A$ the \textit{stabilizer} of $a$ in $G$ is the set of elements of $G$ that fix the element $a$: $\{g\in G \mid g\cdot a = a\}$ and is denoted by $G_a$.
    \item An action is \textit{faithful} if its kernel is the identity.
   \end{enumerate}
 \end{dfn}
 
 
 \begin{nt}
  The kernel pf an action is precisely the same as the kernel of the associated permutation representation as defined in the note in section 1.7 and is rephrased below. 
 \end{nt}
 
 
 \begin{prop}
  For any group $G$ and any nonempty set $A$ there is a bijection between the actions of $G$ on $A$ and the homomorphisms of $G$ into $S_A$.
 \end{prop}
 
 
 \begin{dfn}
  If $G$ is a group a \textit{permutation representation} of $G$ into the symmetric group $S_A$ for some nonempty set $A$. We shall say a given action of $G$ on $A$ \textit{affords} or \textit{induces} the associated representation of $G$. 
 \end{dfn}
 
 
 \begin{prop}
  Let $G$ be a group acting on the nonempty set $A$. the relation on $A$ defined by 
  \[a\sim b \text{ if and only if } a=g\cdot b \text{ for some } g\in G\]
  is an equivalence relation. For each $a\in A$, the number of elements in the equivalence class containing $a$ is $|G:G_a|$, the index of the stabilizer of $a$.
 \end{prop}
 
 
 \begin{dfn}
  Let $G$ be a group acting on the set $A$.
  \begin{enumerate}
   \item The equivalence class $\{g\cdot \mid g\in G\}$ is called the \textit{orbit} of $G$ containing $a$.
   \item The action of $G$ on $A$ is called \textit{transitive} if there is only one orbit, i.e., given any two elements $a,b\in A$ there is some $g\in G$ such that $a=g\cdot b$.
  \end{enumerate}
 \end{dfn}
 
 
 \begin{nt}
  ~\begin{enumerate}
    \item Every element of $S_n$ has a unique cycle decomposition
    \item Subgroups of symmetric groups are called \textit{permutation groups}.
    \item The orbits of a permutation group will refer to its orbits on $\{1,2,\ldots,n\}$
    \item The orbits of an element $\sigma \in S_n$ will refer to the orbits of the group $\langle \sigma \rangle$.
   \end{enumerate}
 \end{nt}
 
 
 \subsection{Group Acting on Themselves by Left Multiplication - Cayley's Theorem}
 
 
 \begin{thm}
  
 \end{thm}





 
 
 
 
 
 
 
 
 
 
 
 
 
 
 
 
 
 
 
 
 
 
 
 
 
 
 
 
 
 
 
\end{document}
