\documentclass[../main]{subfiles}


\begin{document}
 
 
 \section{Group Actions}
 
 
 \subsection{Group Actions and Permutation Representations}
 
 
 \begin{dfn} Let $G$ be a group acting on a set $A$
  \begin{enumerate}
    \item The \textit{kernel} of the action is the set of elements of $G$ that act trivially on every element of $A$: $\{g\in G \mid g \cdot a = a $ for all $a\in A\}$.
    \item For each $a\in A$ the \textit{stabilizer} of $a$ in $G$ is the set of elements of $G$ that fix the element $a$: $\{g\in G \mid g\cdot a = a\}$ and is denoted by $G_a$.
    \item An action is \textit{faithful} if its kernel is the identity.
   \end{enumerate}
 \end{dfn}
 
 
 \begin{nt}
  The kernel pf an action is precisely the same as the kernel of the associated permutation representation as defined in the note in section 1.7 and is rephrased below. 
 \end{nt}
 
 
 \begin{prop}
  For any group $G$ and any nonempty set $A$ there is a bijection between the actions of $G$ on $A$ and the homomorphisms of $G$ into $S_A$.
 \end{prop}
 
 
 \begin{dfn}
  If $G$ is a group a \textit{permutation representation} of $G$ into the symmetric group $S_A$ for some nonempty set $A$. We shall say a given action of $G$ on $A$ \textit{affords} or \textit{induces} the associated representation of $G$. 
 \end{dfn}
 
 
 \begin{prop}
  Let $G$ be a group acting on the nonempty set $A$. the relation on $A$ defined by 
  \[a\sim b \text{ if and only if } a=g\cdot b \text{ for some } g\in G\]
  is an equivalence relation. For each $a\in A$, the number of elements in the equivalence class containing $a$ is $|G:G_a|$, the index of the stabilizer of $a$.
 \end{prop}
 
 
 \begin{dfn}
  Let $G$ be a group acting on the set $A$.
  \begin{enumerate}
   \item The equivalence class $\{g\cdot a \mid g\in G\}$ is called the \textit{orbit} of $G$ containing $a$.
   \item The action of $G$ on $A$ is called \textit{transitive} if there is only one orbit, i.e., given any two elements $a,b\in A$ there is some $g\in G$ such that $a=g\cdot b$.
  \end{enumerate}
 \end{dfn}
 
 
 \begin{nt}
  ~\begin{enumerate}
    \item Every element of $S_n$ has a unique cycle decomposition
    \item Subgroups of symmetric groups are called \textit{permutation groups}.
    \item The orbits of a permutation group will refer to its orbits on $\{1,2,\ldots,n\}$
    \item The orbits of an element $\sigma \in S_n$ will refer to the orbits of the group $\langle \sigma \rangle$.
   \end{enumerate}
 \end{nt}
 
 
 \subsection{Group Acting on Themselves by Left Multiplication - Cayley's Theorem}
 

 \begin{nt}
  In this section $G$ is any group and we first consider $G$ acting on itself (i.e., $A=G$) by left multiplication:
  \[g \cdot a=g a \quad \text { for all } g \in G, a \in G\]

When $G$ is a finite group of order $n$ it is convenient to label the elements of $G$ with the integers $1,2, \ldots, n$ in order to describe the permutation representation afforded by this action. In this way the elements of $G$ are listed as $g_1, g_2, \ldots, g_n$ and for each $g \in G$ the permutation $\sigma_g$ may be described as a permutation of the indices $1,2, \ldots, n$ as follows:
\[\sigma_g(i)=j \quad \text { if and only if } \quad g g_i=g_j.\]
 \end{nt}
 
 
 \begin{thm}
  Let $G$ be a group, let $H$ be a subgroup and let $G$ act by left multiplication on the set $A$ of left cosets of $H$ in $G$. Let $\pi_H$ be the associated permutation representation afforded by this action. Then 
  \begin{enumerate}
   \item $G$ acts transitively on $A$
   \item the stabilizer of $G$ of the point $1H\in A$ us the subgroup $H$
   \item the kernel of the action (i.e., the kernel of $\pi_H$) is $\cap_{x\in G} xHx^{-1}$, and ker$\pi_H$ is the largest normal subgroup of $g$ contained in $H$.
  \end{enumerate}
 \end{thm}
 
 
 \begin{cor}
  (Cayley's Theorem) Every group is isomorphic to a subgroup of symmetric group. If $G$ is a group of order $n$, then $G$ is isomorphic to a subgroup of $S_n$.
 \end{cor}
 
 
 \begin{cor}
  If $G$ is a finite group of order $n$ and $p$ is the smallest prime dividing $|G|$, then any subgroup of index $p$ is normal (Note that a group of order $n$ need not have a subgroup of order $p$).
 \end{cor}
 
 
 \subsection{Groups Acting on Themselves by Conjugation - The Class Equation}
 
 
 \begin{nt}
  In this section we consider a group $G$ acting on itself by \textit{conjugation}
  \[g\cdot a = gag^{-1} \quad \text{for all } g\in G, a\in G\]
 \end{nt}
 
 
 \begin{dfn}
  Two elements $a$ and $a$ of $G$ are said to be \textit{conjugate} if $G$ if there is some $g\in G$ such that $b = gag^{-1}$ (i.e., if and only if they are in some orbit of $G$ acting on itself by conjugation). The orbits of $G$ acting on itself by conjugation are called \textit{conjugacy classes} of $G$.
 \end{dfn}
 
 
 \begin{dfn}
  Two subsets $S$ and $T$ of $G$ are said to be \textit{conjugate in} $G$ if there is some $g \in G$ such that $T = gSg^{-1}$ (i.e., if and only if they are in the same orbit of $G$ acting on its subsets by conjugation). 
 \end{dfn}
 
 
 \begin{prop}
  The number of conjugates of a subset $S$ in a group $G$ is the index of the normalizer of $S$, $|G:N_G(S)|$. In particular, the number of conjugates of an element $s$ of $G$ is the index of the centralizer of $s$, $|G:C_g(s)|$.
 \end{prop}
 
 
 \begin{thm}
  (The Class Equation) Let $G$ be a finite group and let $g_1,g_2, \ldots ,g_r$ be representatives of the distinct conjugacy classes of $G$ not contained in the center $Z(G)$ of $G$. Then 
  \[|G| = |Z(G)| + \sum_{i=1}^r |G:C_G(g_i)|.\]
 \end{thm}
 
 
 \begin{thm}
  If $p$ is a prime and $P$ is a group of prime order $p^\alpha$ for some $\alpha \geq 1$, then $P$ has a nontrivial center: $Z(P) \neq 1$.
 \end{thm}
 
 
 \begin{prop}
  Let $\sigma,\tau$ be elements of the symmetric group $S_n$ and suppose $\sigma$ has cycle decomposition 
  \[(a_1a_2 \ldots a_{k_1})(b_1 b_2 \ldots b_{k_2})\ldots.\]
  Then $\tau \sigma \tau^{-1}$ has cycle decomposition
  \[(\tau(a_1) \tau(a_2) \ldots \tau(a_{k_1})) (\tau(b_1) \tau(b_2) \ldots \tau(b_{k_2})) \ldots,\]
  that is $\tau \sigma \tau^{-1}$ is obtained from $\sigma$ by replacing each $i$ in the cycle decomposition for $\sigma$ by the entry $\tau(i)$.
 \end{prop}
 
 
 \begin{dfn}
  ~\begin{enumerate}
    \item If $\sigma \in S_n$ is the product of disjoint cycles of length $n_1, n_2, \ldots , n_r$ with $n_1 \leq n_2 \leq \ldots \leq n_r$ (including its 1-cycles) then the integers $n_1, n_2, \ldots , n_r$ are called the \textit{cycle type} of $\sigma$.
    \item If $n \in \Z^+$, a \textit{partition} of $n$ is any nondecreasing sequence of positive integers whose sum is $n$.
   \end{enumerate}
 \end{dfn}
 
 
 \begin{prop}
  Two elements of $S_n$ are conjugate in $S_n$ if and only if they have the same cycle type. The number of conjugacy classes of $S_n$ equals the number of partitions of $n$.
 \end{prop}
 
 
 \begin{thm}
  $A_5$ is a simple group.
 \end{thm}
 
 
 \subsection{Automorphisms}
 
 
 \begin{dfn}
  Let $G$ be a group. An isomorphism from $G$ onto itself is called an \textit{automorphism} of $G$. The set of all automorphisms of $G$ is denoted Aut$(G)$. 
 \end{dfn}
 
 
 \begin{nt}
  Aut$(G)$ is a group under composition.
 \end{nt}
 
 
 \begin{prop}
  Let $H$ be a normal subgroup of the group $G$. Then $G$ acts by conjugation on $H$ as automorphisms of $H$. More specifically, the action of $G$ on $H$ by conjugation is defined for each $g \in G$ by 
  \[ h\mapsto ghg^{-1}\qquad \text{for each } h \in H.\]
  For each $g \in G$, conjugation by $g$ is an automorphism of $H$. The permutation representation afforded by this action is a homomorphism of $G$ into Aut$(H)$ with kernel $C_G(H)$. In particular, $G/C_G(H)$ is isomorphic to a subgroup of Aut$(H)$.
 \end{prop}
 
 
 \begin{cor}
  If $K$ is any subgroup of the group $G$ and $g\in G$, then $K \cong gKg^{-1}$. Conjugate elements and conjugate subgroups have the same order.  
 \end{cor}
 
 
 \begin{cor}
  For any subgroup $H$ of a group $G$, the quotient group $N_G(H)/C_G(H)$ is isomorphic to a subgroup of Aut$(H)$. In particular, $G/Z(G)$ is isomorphic to a subgroup of Aut$(G)$.
 \end{cor}
 
 
 \begin{dfn}
  Let $G$ be a group and let $g \in G$. Conjugation by $g$ is called an \textit{inner automorphism} of $G$ and the subgroup of Aut$(G)$ consisting of all inner automorphisms is denoted Inn$(G)$.
 \end{dfn}
 
 
 \begin{dfn}
  A subgroup $H$ of a group $G$ is called \textit{characteristic} in $G$, denoted $H$ char $G$, if every automorphism of $G$ maps $H$ to itself, i.e., $\sigma(H)=H$ for all $\sigma \in$ Aut$(G)$.
 \end{dfn}
 
 
 \begin{nt}
  ~\begin{enumerate}
    \item Characteristic subgroups are normal,
    \item if $H$ is the unique subgroup of a given order, then $H$ is characteristic in $G$, and
    \item if $K$ char $H$ and $H \nor G$, then $K \nor G$.
   \end{enumerate}
 \end{nt}

 
 \begin{prop}
  The automorphism group of the cyclic group of order $n$ is isomorphic to $(\Z/n\Z)^\times$, an abelian group of order $\phi(n)$ (where $\phi$ is Euler's function).
 \end{prop}
 
 
 \begin{prop}
  ~\begin{enumerate}
    \item If $p$ is an odd prime and $n \in \Z^+$, then the automorphism group of the cyclic group of order $p$ is cyclic of order $p-1$. More generally, the automorphism group of the cyclic grup of order $p^n$ is cyclic of order $p^{n-1}(p-1)$.
    \item For all $n\geq 3$ the automorphism group of the cyclic group of order $2^n$ is isomorphic to $Z_2 \times Z_{2^{n-2}}$, and in particular is not cyclic but has a cyclic subgroup of index 2.
    \item Let $p$ be a prime and let $V$  be an abelian group (written additively)with the property that $pv=0$ for all $v\in V$. If $|V|=p^n$, then $V$ is an $n$-dimensional vector space over the field $\F_p = \Z/p\Z$. The automorphisms of $V$ are precisely the nonsingular linear transformations from $V$ to itself, that is 
    \[\text{Aut}(V) \cong GL(V) \cong GL_n(\F_p).\]
    In particular, the order of Aut$(V)$ is given in section 1.4.
    \item For all $n \neq 6$ we have Aut$(S_n) = Inn(S_n) \cong S_n$. For $n = 6$ we have \\ $|\text{Aut}(S_6):\text{Inn}(S_6)| = 2$.
    \item Aut$(D_8) \cong D_8$ and Aut$(Q_8) \cong S_4$. 
   \end{enumerate}
 \end{prop}
 
 
 \subsection{Sylow's Theorem}
 
 
 \begin{dfn}
  Let $G$ be a group and let $p$ be a prime.
  \begin{enumerate}
   \item A group of order $p^\alpha$ for some $\alpha \geq 0$ is called a \textit{$p$-group}. Subgroups of $G$ which are $p$-groups are called \textit{$p$-subgroups}. 
   \item If $G$ is a group of order $p^\alpha m$, where $p\nmid m$, then a subgroup of order $p^\alpha$ is called a \textit{Sylow $p$-subgroup} of $G$.
   \item The set of Sylow $p$-subgroups of $G$ will be denoted $Syl_p(G)$ and the number of Sylow $p$-subgroups of $G $ will be denoted by $n_p(G)$ (or just $n_p$ when $G$ is clear from context).
  \end{enumerate}
 \end{dfn}
 
 
 \begin{thm}
  (Sylow's Theorem) Let $G$ be a group of order $p^\alpha m$, where $p$ is a prime not dividing m. ;
  \begin{enumerate}
   \item Sylow $p$-subgroups of $G$ exist, i.e., $Syl_p(G) \neq \emptyset$.
   \item If $P$ is a sylow $p$-subgroup of $G$ and $Q$ is any $p$-subgroup of $G$, then there exists $g \in G$ such that $Q \leq gPg^{-1}$, i.e., $Q$ is contained in some conjugate of $P$. In particular, any two  Sylow $p$-subgroups of $G$ are conjugate in $G$.
   \item The number of Sylow $p$-subgroups of $G$ is of the form $1+kp$, i.e., 
   \[n_p=1 \pmod{p}.\]
   Further, $n_p$ is the indec in $G$ of the normalizer of $N_G(P)$ for any Sylow $p$-subgroup $P$, hence $n_p$ divides $m$.
  \end{enumerate}
 \end{thm}
 
 
 \begin{lem}
  Let $P\in Sly_p(G)$. If Q is any $p$-subgroup of $G$, then $Q\cap N_G(P) = Q\cap P$.
 \end{lem}
 
 
 \begin{cor}
  Let $P$ be a Sylow $p$-subgroup of $G$. Then the following are equivalent:
  \begin{enumerate}
   \item $P$ is the unique Sylow $p$-subgroup of $G$, i.e., $n_p = 1$
   \item $P$ is normal in $G$
   \item $P$ is characteristic in $G$ 
   \item All subgroups generated by elements of $p$-power order are $p$-groups, i.e., if $X$ is any subset of $G$ such that $|x|$ is a power of $p$ for all $x\in X$, then $\langle X \rangle$ is a $p$-group.
  \end{enumerate}
 \end{cor}
 
 
 \begin{prop}
  If $|G| = 60$ and $G$ has more than one Sylow $5$-subgroups, then $G$ is simple.
 \end{prop}
 
 
 \begin{cor}
  $A_5$ is simple
 \end{cor}
 
 
 \begin{prop}
  If $G$ is a simple group of order $60$, then $G \cong A_5$. 
 \end{prop}
 
 
 \subsection{The Simplicity of $A_n$}
 
 
 \begin{thm}
  $A_n$ is simple for all $n\geq 5$.
 \end{thm}
 
 
\end{document}
