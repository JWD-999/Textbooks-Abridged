\documentclass[../main]{subfiles}

\begin{document}

\section{Introduction to Rings}


\subsection{Basic Definitions and Examples}


\begin{dfn}
 ~\begin{enumerate}
   \item A \textit{ring} $R$ is a set together with two binary operations $+$ and $\times$ (called addition and multiplication) satisfying the following axioms:
   
   \begin{enumerate}
    \item $(R,+)$ is an abelian group,
    
    \item $\times$ is associative: $(a \times b) \times c = a \times (b \times c)$ for all $a,b,c \in R$,
    
    \item the \textit{distributive laws} hold in $R$: for all $a,b,c \in R$,
    \[ (a+b)\times c = (a\times c)+(b\times c) \quad \text{and} \quad a\times (b+c) = (a\times b)+(a\times c). \]
   \end{enumerate}
   
   \item The ring $R$ is \textit{commutative} if multiplication is commutative.
   
   \item The ring $R$ is said to have an \textit{identity} (or \textit{contain a 1}) if there is an element $1\in R$ with 
   \[ 1\times a = a\times 1 = a \qquad \text{for all } a\in R. \]
  \end{enumerate}
\end{dfn}


\begin{nt}
 ~\begin{enumerate}
  \item We shall write $ab$ rather than $a\times b$ for $a,b\in R$. 
  
  \item The additive identity of $R$ will be denoted by 0
  
  \item The additive of an element $a$ will be denoted $-a$.
 \end{enumerate}
\end{nt}


\begin{nt}
 $R = \{0\}$ is called the \textit{zero ring}, denoted $R = 0$. $R = 0$ is the only ring where $1 = 0$. We will often exclude this ring by imposing the condition $1 \neq 0$.
\end{nt}


\begin{dfn}
 A ring $R$ with identity $1 \neq 0$, is called a \textit{division ring} (or \textit{skew field}) if every nonzero element $a\in R$ has a multiplicative inverse, i.e., there exists $b\in R$ such that $ab = ba = 1$. A commutative division ring is called a \textit{field}.
\end{dfn}


\begin{prop}
 Let $R$ be a ring. Then 
 \begin{enumerate}
  \item $0a = a0 = 0$ for all $a \in R$.
  
  \item $(-a)b = a(-b) = -(ab)$ for all $a,b\in R$.
  
  \item $(-a)(-b) = ab$ for all $a,b\in R$.
  
  \item If $R$ has an identity 1, then the identity is unique and $-a = -1(a)$.
 \end{enumerate}
\end{prop}


\begin{dfn}
 Let $R$ be a ring 
 \begin{enumerate}
  \item A nonzero element $a$ of $R$ is called a \textit{zero divisor} if there is a nonzero element $b$ of $R$ such that either $ab = 0$ or $ba = 0$.
  
  \item Assume $R$ has an identity $1 \neq 0$. An element $u$ of $R$ is called a \textit{unit} in $R$ if there is some $v$ in $R$ such that $vu = uv = 1$. The set of units in $R$ is denoted $R^{\times}$.
 \end{enumerate}
\end{dfn}


\begin{nt}
 ~\begin{enumerate}
   \item $R^{\times}$ forms a group under multiplication and will be referred to as the \textit{group of units} of $R$.
   
   \item Using the above terminology a field is a commutative ring $F$ with identity $1 \neq 0$ in which every nonzero element is a unit, i.e., $F^{\times} = F - \{0\}$.
  \end{enumerate}
\end{nt}


\begin{dfn}
 A commutative ring with identity $1 \neq 0$ is called an \textit{integral domain} if it has no zero divisors.
\end{dfn}


\begin{prop}
 Assume $a,b$ and $c$ are elements of any ring with $a$ not a zero divisor. If $ab = ac$ then either $a = 0$ or $b = c$ (i.e., if $a \neq 0$ we can cancel the $a$'s). In particular, if $a,b,c$ are elements in an integral domain and $ab = ac$, then either $a = 0$ or $b = c$.
\end{prop}


\begin{cor}
 Any finite integral domain is a field.
\end{cor}


\begin{dfn}
 A \textit{subring} of the ring $R$ is a subgroup of $R$ that is closed under multiplication.
\end{dfn}


\begin{nt}
 To show that a subset of a ring $R$ is a subring it is enough to show that it is nonempty and closed under subtraction and under multiplication.
\end{nt}


\subsection{Examples: Polynomial Rings, Matrix Rings, and Group Rings}

% Might want to add the construction of these rings'


\begin{prop}
 Let $R$ be an integral domain and let $p(x), q(x)$ be nonzero elements of $R[x]$. Then 
 \begin{enumerate}
  \item $\text{degree} p(x) q(x) = \text{degree} p(x) + \text{degree} q(x)$,
  
  \item The units of $R[x]$ are just the units of $R$,
  
  \item $R[x]$ is an integral domain. 
 \end{enumerate}
\end{prop}


\subsection{Ring Homomorphisms and Quotient Rings}


\begin{dfn}
 Let $R$ and $S$ be rings.
 \begin{enumerate}
  \item A \textit{ring homomorphism} is a map $\phi \colon R \to S$ satisfying
  
  \begin{enumerate}
   \item $\phi(a + b) = \phi(a) + \phi(b)$ for all $a,b \in R$ (so $\phi$ is a group homomorphism on the additive groups) and 
   
   \item $\phi(ab) = \phi(a)\phi(b)$ for all $a,b \in R$.
  \end{enumerate}
  
  \item The \textit{kernel} of the ring homomorphism $\phi$, denoted ker$\phi$, is the set of elements of $R$ that map to 0 in $S$. (i.e., the kernel of $\phi$ viewed as a homomorphism of additive groups).
  
  \item A bijective ring homomorphism is called an \textit{isomorphism}.
 \end{enumerate}
\end{dfn}


\begin{prop}
 Let $R$ and $S$ be rings and let $\phi \colon R \to S$ be a homomorphism.
 \begin{enumerate}
  \item The image of $\phi$ is a subring of $S$.
  
  \item The kernel of $\phi$ is a subring of $R$. Furthermore, if $\alpha \in$ker$\phi$ then $r\alpha$ and $\alpha r \in$ker$\phi$ for every $r\in R$, i.e., ker$\phi$ is closed under multiplication by elements from $R$.
 \end{enumerate}
\end{prop}


\begin{dfn}
 Let $R$ be a ring, let $I$ be a subset of $R$ and let $r \in R$.
 \begin{enumerate}
  \item $rI = \{ra \mid a\in I\} \quad \text{and} \quad Ir = \{ar \mid a\in I\}$.
  
  \item A subset $I$ of $R$ is a \textit{left Ideal} of $R$ if 
  \begin{enumerate}
   \item $I$ is a subring of $R$, and 
   
   \item $I$ is closed under left multiplication by elements of $R$, i.e., $rI \subseteq I$ for all $r\in R$.
  \end{enumerate}
  
  Similarly $I$ is a \textit{right ideal} if (a) holds and in place of (b) one has 
  
  \begin{enumerate}[label = {(\alph*)'}]
  \setcounter{enumii}{1}
   \item $I$ is closed under right multiplication by elements from $R$, i.e., $Ir \subseteq I$ for all $r \in R$.
  \end{enumerate}
  
  \item A subset $I$ that is both a left ideal and a right ideal is called an \textit{ideal} (or, for added emphasis, a \textit{two-sided ideal}) of $R$.
 \end{enumerate}
\end{dfn}


\begin{prop}
 Let $R$ be a ring and let $I$ be an ideal of $R$. Then the (additive) quotient group $R/I$ is a ring under the binary operations:
 \[ (r + I) + (s + I) = (r+s)+I \qquad \textit{and} \qquad (r+I)\times (s+I) = (rs) +I \]
 for all $r,s \in R$. Conversely, if $I$ is any subgroup such that the above operations are well defined, then $I$ is an ideal of $R$. 
\end{prop}


\begin{dfn}
 When $I$ is an ideal of $R$ the ring $R/I$ with the operations in the previous proposition us called the \textit{quotient ring} of $R$ by $I$.
\end{dfn}


\begin{thm}
 \begin{enumerate}
  \item (The First Isomorphism Theorem for Rings) If $\phi \colon R \to S$ is a homomorphism of rings, then the kernel of $\phi$ is an ideal of $R$, the image of $\phi$ is a subring of $S$ and $R/$ker$\phi$ is isomorphic as a ring to $\phi(R)$.
  
  \item If $I$ is any ideal of $R$, then the map 
  \[ R \to R/I \qquad \text{defined by} \qquad r \mapsto r+I \]
  is a surjective ring homomorphism with kernel $I$ (this homomorphism is called the \textit{natural projection} of $R$ onto $R/I$). Thus every ideal is the kernel of a ring homomorphism and vice versa. 
 \end{enumerate}
\end{thm}


\begin{thm}
 Let $R$ be a ring.
 \begin{enumerate}
  \item (The Second Isomorphism Theorem for Rings) Let $A$ be a subring and let $B$ be an ideal of $R$. Then $A+B = \{a+b \mid a\in A, b\in B \}$ is a subring of $R$, $A \cap B$ is an ideal of $A$ and $(A+B)/B \cong A/(A \cap B)$.
  
  \item (The Third Isomorphism Theorem for Rings) Let $I$ and $J$ be ideals of $R$ with $I \subseteq J$. Then $J/I$ is an ideal of $R/I$ and $(R/I)/(J/I) \cong R/J$.
  
  \item (The Fourth or Lattice Isomorphism Theorem for Rings) Let $I$ be an ideal of $R$. The correspondence $A \leftrightarrow A/I$ is an inclusion preserving bijective between the set of subrings $A$ of $R$ that contain $I$ and the set of subrings of $R/I$. Furthermore, $A$ (a subring containing $I$) is an ideal of $R$ if and only if $A/I$ is an ideal of $R/I$.
 \end{enumerate}
\end{thm}


\begin{dfn}
 Let $I$ and $J$ be ideals of $R$. 
 \begin{enumerate}
  \item Define the \textit{sum} of $I$ and $J$ by $I+J = \{ a+b \mid a\in I, b\in J \}$.
  
  \item Define the \textit{product} of $I$ and $J$, denoted by $IJ$, to be the set of all finite sums of elements of the form $ab$ with $a \in I$ and $b \in J$.
  
  \item For any $n\geq 1$, define the \textit{$n^{\text{th}}$ power} of $I$, denoted $I^n$, to be the set consisting of all finite sums of elements of the form $a_1 a_2 \cdots a_n$ with $a_i \in I$ for all $i$. Equivalently, $I^n$ is defined inductively by defining $I^1 = I$ and $I^n = II^{n-1}$ for $n = 2, 3, \ldots$. 
 \end{enumerate}
\end{dfn}


\subsection{Properties of Ideals}

Throughout this section $R$ is a ring with identity $1 \neq 0$.


\begin{dfn}
 
\end{dfn}




















 
 
 
 
 
 
 
 
 
 
 
 
 
 
 
 
 
 
 
 
 
 
 
 
 
\end{document}
