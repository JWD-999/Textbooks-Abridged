\documentclass[../main]{subfiles}

\begin{document}

\section{Introduction to Rings}


\subsection{Basic Definitions and Examples}


\begin{dfn}
 ~\begin{enumerate}
   \item A \textit{ring} $R$ is a set together with two binary operations $+$ and $\times$ (called addition and multiplication) satisfying the following axioms:
   
   \begin{enumerate}
    \item $(R,+)$ is an abelian group,
    
    \item $\times$ is associative: $(a \times b) \times c = a \times (b \times c)$ for all $a,b,c \in R$,
    
    \item the \textit{distributive laws} hold in $R$: for all $a,b,c \in R$,
    \[ (a+b)\times c = (a\times c)+(b\times c) \quad \text{and} \quad a\times (b+c) = (a\times b)+(a\times c). \]
   \end{enumerate}
   
   \item The ring $R$ is \textit{commutative} if multiplication is commutative.
   
   \item The ring $R$ is said to have an \textit{identity} (or \textit{contain a 1}) if there is an element $1\in R$ with 
   \[ 1\times a = a\times 1 = a \qquad \text{for all } a\in R. \]
  \end{enumerate}
\end{dfn}


\begin{nt}
 ~\begin{enumerate}
  \item We shall write $ab$ rather than $a\times b$ for $a,b\in R$. 
  
  \item The additive identity of $R$ will be denoted by 0
  
  \item The additive of an element $a$ will be denoted $-a$.
 \end{enumerate}
\end{nt}


\begin{nt}
 $R = \{0\}$ is called the \textit{zero ring}, denoted $R = 0$. $R = 0$ is the only ring where $1 = 0$. We will often exclude this ring by imposing the condition $1 \neq 0$.
\end{nt}


\begin{dfn}
 A ring $R$ with identity $1 \neq 0$, is called a \textit{division ring} (or \textit{skew field}) if every nonzero element $a\in R$ has a multiplicative inverse, i.e., there exists $b\in R$ such that $ab = ba = 1$. A commutative division ring is called a \textit{field}.
\end{dfn}


\begin{prop}
 Let $R$ be a ring. Then 
 \begin{enumerate}
  \item $0a = a0 = 0$ for all $a \in R$.
  
  \item $(-a)b = a(-b) = -(ab)$ for all $a,b\in R$.
  
  \item $(-a)(-b) = ab$ for all $a,b\in R$.
  
  \item If $R$ has an identity 1, then the identity is unique and $-a = -1(a)$.
 \end{enumerate}
\end{prop}


\begin{dfn}
 Let $R$ be a ring 
 \begin{enumerate}
  \item A nonzero element $a$ of $R$ is called a \textit{zero divisor} if there is a nonzero element $b$ of $R$ such that either $ab = 0$ or $ba = 0$.
  
  \item Assume $R$ has an identity $1 \neq 0$. An element $u$ of $R$ is called a \textit{unit} in $R$ if there is some $v$ in $R$ such that $vu = uv = 1$. The set of units in $R$ is denoted $R^{\times}$.
 \end{enumerate}
\end{dfn}


\begin{nt}
 ~\begin{enumerate}
   \item $R^{\times}$ forms a group under multiplication and will be referred to as the \textit{group of units} of $R$.
   
   \item Using the above terminology a field is a commutative ring $F$ with identity $1 \neq 0$ in which every nonzero element is a unit, i.e., $F^{\times} = F - \{0\}$.
  \end{enumerate}
\end{nt}


\begin{dfn}
 A commutative ring with identity $1 \neq 0$ is called an \textit{integral domain} if it has no zero divisors.
\end{dfn}


\begin{prop}
 Assume $a,b$ and $c$ are elements of any ring with $a$ not a zero divisor. If $ab = ac$ then either $a = 0$ or $b = c$ (i.e., if $a \neq 0$ we can cancel the $a$'s). In particular, if $a,b,c$ are elements in an integral domain and $ab = ac$, then either $a = 0$ or $b = c$.
\end{prop}


\begin{cor}
 Any finite integral domain is a field.
\end{cor}


\begin{dfn}
 A \textit{subring} of the ring $R$ is a subgroup of $R$ that is closed under multiplication.
\end{dfn}


\begin{nt}
 To show that a subset of a ring $R$ is a subring it is enough to show that it is nonempty and closed under subtraction and under multiplication.
\end{nt}


\subsection{Examples: Polynomial Rings, Matrix Rings, and Group Rings}
%might skip










 
 
 
 
 
 
 
 
 
 
 
 
 
 
 
 
 
 
 
 
 
 
 
 
 
\end{document}
