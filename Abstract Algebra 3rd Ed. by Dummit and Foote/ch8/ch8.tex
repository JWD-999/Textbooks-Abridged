\documentclass[../main]{subfiles}

\begin{document}
 
\section{Euclidean Domains, Principal Ideal Domains, and Unique Factorization Domains}


All rings in this chapter are commutative

\subsection{Euclidean Domains}


\begin{dfn}
 Any function $N \colon R \to \Z^+ \cup \{0\}$ with $N(0) = 0$ is called a \textit{norm} on the integral domain $R$. If $N(a) > 0$ for $a\neq 0$ define $N$ to be a \textit{positive norm}.
\end{dfn}


\begin{dfn}
 The integral domain $R$ is said to be a \textit{Euclidean Domain} (or possess a \textit{Division Algorithm}) if there is a norm $N$ on $R$ such that for any two elements $a$ and $b$ of $R$ with $b \neq 0$ there exist elements $q$ and $r$ in $R$ with
 \[ a=qb+r \quad \text{with } r=0 \text{ or } N(r) < N(b). \]
 The element $q$ is called the \textit{quotient} and the element $r$ the \textit{remainder} of the division.
\end{dfn}


\begin{prop}
 Every ideal in a Euclidean Domain is principal. More precisely, if $I$ is any nonzero ideal in the Euclidean Domain $R$ then $I = (d)$, where $d$ is any nonzero element of $I$ of minimal norm.
\end{prop}


\begin{dfn}
 Let $R$ be a commutative ring and let $a,b \in R$ with $b \neq 0$.
 \begin{enumerate}
  \item $a$ is said to be a \textit{multiple} of $b$ if there exists an element $x \in R$ with $a=bx$. In this case $b$ is said to \textit{divide} $a$ or be a divisor of $a$, written $b|a$.
  
  \item A \textit{greatest common divisor} of $a$ and $b$ is a nonzero element $d$ such that
  \begin{enumerate}
   \item $d|a$ and $d|b$, and
   
   \item if $d'|a$ and $d'|b$ then $d'|d$.
  \end{enumerate}
  A greatest common divisor of $a$ and $b$ will be denoted by g.c.d$(a,b)$, or (abusing the notation) simply $(a,b)$
 \end{enumerate}
\end{dfn}


\begin{nt}
 ~\begin{enumerate}
   \item $b|a$ in $R$ if and only if $a \in (b)$ if and only if $(a) \subseteq (b)$.
   
   \item The above definition of greatest common divisor can be restated in terms of ideals as such. If $I$ is the ideal of $R$ generated by $a$ and $b$, then $d$ is a greatest common divisor  of $a$ and $b$ if
   \begin{enumerate}
    \item $I$ is contained in the principal ideal $(d)$, and
    
    \item if $(d')$ is any principal ideal containing $I$  then $(d) \subseteq (d')$.
   \end{enumerate}
  \end{enumerate}
\end{nt}


\begin{prop}
 If $a$ and $b$ are nonzero elements in the commutative ring $R$ such that the ideal generated by $a$ and $b$ is a principal ideal $(d)$, then $d$ is a greatest common divisor of $a$ and $b$.
\end{prop}


\begin{prop}
 Let $R$ be an integral domain. If two elements $d$ and $d'$ of $R$ generate the same principal ideal, i.e., $(d) = (d')$, then $d'=ud$ for some unit $u$ in $R$. In particular, if $d$ and $d'$ are both greatest common divisors of $a$ and $b$, then $d' = ud$ for some unit $u$.
\end{prop}


\begin{thm}
 Let $R$ be a Euclidean Domain and let $a$ and $b$ be nonzero elements of $R$. Let $d=r_n$ be the last nonzero remainder in the Euclidean Algorithm for $a$ and $b$. Then
 \begin{enumerate}
  \item $d$ is a greatest common divisor of $a$ and $b$, and 
  
  \item the principal ideal $(d)$ is the ideal generated by $a$ and $b$. In particular, $d$ can be written as an $R$-linear combination of $a$ and $b$, i.e., there are elements $x$ and $y$ in $R$ such that 
  \[ d = ax + by. \]
 \end{enumerate}
\end{thm}


\addtocounter{thm}{1}


\subsection{Principal Ideal Domains (P.I.D.s)}


\begin{dfn}
 A \textit{Principal Ideal Domain} (P.I.D) is an integral domain in which every ideal is principal.
\end{dfn}


\begin{nt}
 By Proposition 1 every Euclidean Domain is a Principal Ideal Domain. So every result about P.I.D.s automatically holds for Euclidean Domains.
\end{nt}


\begin{prop}
 Let $R$ be a Principal Ideal Domain and let $a$ and $b$ be nonzero elements of $R$. Let $d$ be a generator for the principal ideal generated by $a$ and $b$. Then 
 \begin{enumerate}
  \item $d$ is a greatest common divisor of $a$ and $b$
  
  \item $d$ can be written as an $R$-linear combination of $a$ and $b$
  
  \item $d$ is unique up to multiplication by a unit of $R$.
 \end{enumerate}
\end{prop}


\begin{prop}
 Every nonzero prime ideal in a Principal Ideal Domain is a maximal ideal.
\end{prop}


\begin{cor}
 If $R$ is any commutative ring such that the ring $R[x]$ is a Principal Ideal Domain (or Euclidean Domain), then $R$ is necessarily a field. 
\end{cor}


\begin{dfn}
 Define $N$ to be a \textit{Dedekind-Hasse norm} if $N$ is a positive norm and for every nonzero $a,b \in R$ either $a$ is an element of the ideal $(b)$ or there is a nonzero element of the ideal $(a,b)$ of norm strictly smaller then the norm of b (i.e., either $b$ divides $a$ in $R$ or there exist $s,t \in R$ with $0 < N(sa - tb) < N(b)$).
\end{dfn}


\begin{prop}
 The integral domain $R$ is a P.I.D if and only if $R$ has a Dedekind-Hasse norm.
\end{prop}


\subsection{Unique Factorization Domains (U.F.D.s)}


\begin{dfn}
 Let $R$ be an integral domain.
 \begin{enumerate}
  \item Suppose $r \in R$ is nonzero and is not a unit. Then r is called \textit{irreducible} in $R$ if whenever $r = ab$ with $a,b\in R$, at least one of $a$ or $b$ must be a unit in $R$. Otherwise $r$ is said to be \textit{reducible}.
  
  \item The nonzero element $p \in R$ is called \textit{prime} in $R$ if the ideal $(p)$ generated by $p$ is a prime ideal. In other words, a nonzero $p$ is prime if it is not a unit and whenever $p|ab$ for any $a,b \in R$, then either $p|a$ or $p|b$.
  
  \item Two elements $a$ and $b$ of $R$ differing by a unit are said to be \textit{associate} in $R$ (i.e., $a=ub$ for some unit $u$ in $R$).
 \end{enumerate}
\end{dfn}


\begin{prop}
 In an integral domain a prime element is always irreducible.
\end{prop}


\begin{prop}
 In a Principal Ideal Domain a nonzero element is a prime if and only if it is irreducible.
\end{prop}


\begin{dfn}
 A \textit{Unique Factorization Domain} (U.F.D.) is an integral domain $R$ in which every nonzero element $r \in R$ which is not a unit has the following two properties:
 \begin{enumerate}
  \item $r$ can be written as a finite product of irreducibles $p_i$ in $R$ (not necessarily distinct): $r=p_1 p_2 \cdots p_n$ and 
  
  \item the decomposition in 1. is unique up to associates: namely if $r = q_1 q_2 \cdots q_m$ is another factorization of $r$ into irreducibles, then $m = n$ and there is some renumbering of factors so that $p_i$ is associate to $q_i$ for $i = 1, 2, \ldots, n$. 
 \end{enumerate}
\end{dfn}

\begin{prop}
 In a Unique Factorization Domain a nonzero element is a prime if and only if it is irreducible.
\end{prop}


\begin{prop}
 Let $a$ and $b$ be two nonzero elements of the Unique Factorization Domain $R$ and suppose
 \[ a = u p_1^{e_1} p_2^{e_2} \cdots p_n^{e_n} \qquad \text{and} \qquad b = v p_1^{f_1} p_2^{f_2} \cdots p_n^{f_n} \]
 are prime factorizations for $a$ and $b$, where $u$ and $v$ are units and the primes $p_1, p_2, \ldots, p_n$ are distinct and the exponents $e_i$ and $f_i$ are $\geq 0$. Then the element 
 \[ d = p_1^{min(e_1, f_1)} p_2^{min(e_2, f_2)} \cdots p_n^{min(e_n, f_n)} \]
 (where $d=1$ if all exponents are 0) is the greatest common divisor of $a$ and $b$.
\end{prop}


\begin{thm}
 Every Principal Ideal Domain is a Unique Factorization Domain. In particular, every Euclidean Domain is a Unique Factorization Domain.
\end{thm}


\begin{cor}
 (Fundamental Theorem of Arithmetic) The integers $\Z$ are a Unique Factorization Domain.
\end{cor}


\begin{cor}
 Let $R$ be a P.I.D. Then there exists a multiplicative Dedekind-Hasse norm on $R$.
\end{cor}


\addtocounter{thm}{3}


\begin{nt}
 We have the following inclusions among classes of commutative rings with identity:
 \[ fields \subset Euclidean \; Domains \subset P.I.D.s \subset U.F.D.s \subset integral \; domains \]
 with all containments being proper.
\end{nt}

\end{document}
