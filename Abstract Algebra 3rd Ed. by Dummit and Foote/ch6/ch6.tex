\documentclass[../main]{subfiles}

\begin{document}

\section{Further Topics in Group Theory}


\subsection{$p$-Groups, Nilpotent Groups, and Solvable Groups}


\begin{dfn}
 A \textit{maximal subgroup} of a group $G$ is a proper subgroup $M$ of $G$ such that there is no subgroups $H$ of $G$ with $M < H < G$.
\end{dfn}


\begin{thm}
 Let $p$ be a prime and let $P$ be a group of order $p^a$, $a\geq 1$. Then
 \begin{enumerate}
  \item The center of $P$ is nontrivial: $Z(P) \neq 1$.
  
  \item If $H$ is a nontrivial normal subgroup of $P$ then $H$ contains a subgroup of order $p^b$ that is normal in $P$ for each divisor $p^b$ of $|H|$. In particular, $P$ has a normal subgroup of order $p^b$ for every $b \in \{0,1, \ldots , a\}$.'
  
  \item If $H < P$ then $H < N_P(H)$ (i.e., every proper subgroup of $P$ is a proper subgroup of its normalizer in $P$).
  
  \item Every maximal subgroup of $P$ is of index $p$ and is normal in $P$. 
 \end{enumerate}
\end{thm}


\begin{dfn}
 ~\begin{enumerate}
   \item For any (finite or infinite) group $G$ define the following subgroups inductively:
   \[ Z_0(G) = 1 \qquad Z_1(G) = Z(G) \]
   and $Z_{i+1}(G)$ is the subgroup of $G$ containing $Z_i(G)$ such that 
   \[ Z_{i+1}(G)/Z_i(G) = Z(G/Z_i(G)) \]
   (i.e., $Z_{i+1}(G)$ is the complete preimage in $G$ of the center of $G/Z_i(G)$ under the natural projection). The chain of subgroups
   \[ Z_0(G) \leq Z_1(G) \leq Z_2(G) \leq \ldots \]
   is called the \textit{upper central series of $G$}. (The use of the term ``upper'' indicates that $Z_i(G) \leq Z_{i+1}(G)$.)
   
   \item A group $G$ is called \textit{nilpotent} if $Z_c(G) = G$ for some $c\in \Z$. The smallest $c$ is called the \textit{nilpotence class} of $G$.
  \end{enumerate}
\end{dfn}


\begin{nt}
 ~\begin{enumerate}
   \item If $G$ is abelian then it is nilpotent since $G = Z(G) =Z_1(G)$.
   
   \item The following containments are proper
   \[ \text{cyclic groups} \subset \text{abelian groups} \subset \text{nilpotent groups} \subset \text{solvable groups} \subset \text{all groups} \]
   
   \item For any finite group there must, by order considerations, be an integer $n$ such that
   \[ Z_n(G) = Z_{n+1} = Z_{n+2} = \cdots. \]
   
   \item For infinite groups $G$ it may happen that all $Z_i(G)$ are proper subgroups of $G$ (so $G$ is not nilpotent) but 
   \[ G = \bigcup_{i=0}^\infty Z_i(G). \]
  \end{enumerate}
\end{nt}


\begin{prop}
 Let $p$ be a prime and let $P$ be a group of order $p^a$. Then $P$ is nilpotent of nilpotence class at most $a-1$ for all $a\geq 2$ (and class equal to $a$ when $a = 0$ or 1).
\end{prop}


\begin{thm}
 Let $G$ be a finite group, let $p_1, p_2, \ldots, p_s$ be the distinct primes dividing its order and let $P_i \in Syl_{p_i}(G), 1\leq i \leq s$. Then the following are equivalent:
 \begin{enumerate}
  \item $G$ is nilpotent 
  
  \item if $H < G$ then $H < N_G(H)$, i.e., every proper subgroup of $G$ is a proper subgroup of its normalizer in $G$
  
  \item $P_i \nor G$ for $1 \leq i \leq s$, i.e., every Sylow subgroup is normal in $G$
  
  \item $G \cong P_1 \times P_2 \times \cdots \times P_s$.
 \end{enumerate}
\end{thm}


\begin{cor}
 A finite abelian group is the direct product of its Sylow subgroups. 
\end{cor}


\begin{prop}
 If $G$ is a finite group such that for all positive integers $n$ dividing its order, $G$ contains at most $n$ elements $x$ satisfying $x^n = 1$, then $G$ is cyclic.
\end{prop}


\begin{prop}
 (Frattini's Argument) Let $G$ be a finite group, let $H$ be a normal subgroup of $G$ and let $P$ be a Sylow $p$-subgroup of $H$. Then $G = H N_G(P)$ and $|G : H|$ divides $|N_G(P)|$.
\end{prop}


\begin{prop}
 A finite group is nilpotent if and only if every maximal subgroup is normal. 
\end{prop}


\begin{dfn}
 For any (finite or infinite) group $G$ define the following subgroups inductively:
 \[ G^0 = G, \qquad G^1 = [G,G] \quad \text{and} \quad G^{i+1} = [G,G^i]. \]
 The chain of groups
 \[ G^0 \geq G^1 \geq G^2 \geq \ldots \]
 is called the \textit{lower central series of $G$}. (The term ``lower'' indicates that $G^i \geq G^{i+1}$.)
\end{dfn}


\begin{thm}
 A group $G$ is nilpotent if and only if $G^n = 1$ for some $n \geq 0$. More precisely, $G$ is nilpotent of class $c$ if and only if $c$ is the smallest nonnegative integer such that $G^c = 1$. If $G$ is nilpotent of class $c$ then 
 \[ G^{c-i} \leq Z_i(G) \quad \text{for all } i \in \{0,1,2, \ldots, c\}. \]
\end{thm}


\begin{nt}
 ~\begin{enumerate}
   \item If $G$ is abelian, we have $G' = G^1 = 1$
   
   \item If $G$ is a finite group there must, by order considerations, be an integer $n$ such that 
   \[ G^n = G^{n+1} = G^{n+2} = \cdots. \]
  \end{enumerate}
\end{nt}


\begin{dfn}
 For any group $G$ define the following sequence of subgroups inductively:
 \[ G^{(0)} = G, \qquad G^{(1)} = [G,G], \quad \text{and} \quad G^{(i+1)} = [G^{(i)}, G^{(i)}] \quad \text{for all } i \geq 1. \]
 This series of subgroups is called the \textit{derived} or \textit{commutator series} of $G$. 
\end{dfn}


\begin{thm}
 A group $G$ is solvable if and only if $G^{(n)} = 1$ for some $n \geq 0$. 
\end{thm}


\begin{prop}
 Let $G$ and $K$ be groups, let $H$ be a subgroup of $G$ and let $\phi \colon G \to K$ be a surjective homomorphism.
 \begin{enumerate}
  \item $H^{(i)} \leq G^{(i)}$ for all $i \geq 0$. In particular, if $G$ is solvable, then so is $H$, i.e., subgroups of solvable groups are solvable (and the solvable length  of $H$ is less than or equal to the solvable length of $G$).
  
  \item $\phi(G^{(i)}) = K^{(i)}$. In particular, homomorphic images and quotient groups of solvable groups are solvable (of solvable length less than or equal to that of the domain group).
  
  \item If $N$ is normal in $G$ and both $N$ and $G/N$ are solvable then so is $G$. 
 \end{enumerate}
\end{prop}


\begin{thm}
 Let $G$ be a finite group. 
 \begin{enumerate}
  \item (Burnside) If $|G| = p^a q^b$ for some primes $p$ and $g$, then $G$ is solvable.
  
  \item (Philip Hall) If for every prime $p$ dividing $|G|$ we factor the order of $G$ as $|G| = p^a m$ where $(p,m) = 1$, and $G$ has a subgroup of order $m$, then $G$ is solvable (i.e., if for all primes $p$, $G$ has a subgroup whose index equals the order of a Sylow $p$-subgroup, then $G$ is solvable --- such subgroups are called Sylow $p$-complements).
  
  \item (Feit-Thompson) If $|G|$ is odd then $G$ is solvable.
  
  \item (Thompson) If for every pair of elements $x,y \in G$, $\langle x,y \rangle$ is a solvable group, then $G$ is solvable.
 \end{enumerate}
\end{thm}


\subsection{Applications in Groups of Medium Order}


\begin{prop}
 ~\begin{enumerate}
   \item If $G$ has no subgroup of index 2 and $G \leq S_k$, then $G \leq A_k$.
   
   \item If $P \in Syl_p(S_k)$ for some odd prime $p$, then $P \in Syl_p(A_k)$ and $|N_{A_k}(P)| = \frac{1}{2} |N_{S_k}(P)|$.
  \end{enumerate}
\end{prop}


\begin{lem}
 In a finite group $G$ is $n_p \not\equiv 1 \pmod{p^2}$, then there are distinct Sylow $p$-subgroups $P$ and $R$ of $G$ such that $P \cap R$ is of index $p$ in both $P$ and $R$ (hence is normal in each).
\end{lem}


\addtocounter{thm}{2}


\subsection{A word on Free Groups}


\begin{nt}
 The way that a free group is defined is a bit involved and can be read on page 216
\end{nt}


\begin{thm}
 $F(S)$ is a group under the binary operation defined on page 216.
\end{thm}


\begin{thm}
 Let $G$ be a group, $S$ a set and $\phi \colon S \to G$ a set map. Then there is a unique group homomorphism $\Phi \colon F(S) \to G$ such that the following diagram commutes:
 
 \[\begin{tikzcd}
	S && {F(S)} \\
	&& G
	\arrow["{\text{inclusion}}", from=1-1, to=1-3]
	\arrow["\phi"', from=1-1, to=2-3]
	\arrow["\Phi", from=1-3, to=2-3]
 \end{tikzcd}\]
\end{thm}


\begin{cor}
 $F(S)$ is unique up to a unique isomorphism which is the identity map on the set $S$.
\end{cor}


\begin{dfn}
 The group $F(S)$ is called the \textit{free group} on the set $S$. A group $F$ is a \textit{free group} if there is some set $S$ such that $F = F(S)$ --- in this case we call $S$ a set of \textit{free generators} (or a \textit{free basis}) of $F$. The cardinality of $S$ is called the \textit{rank} of the free group. 
\end{dfn}


\begin{thm}
 (Schreier) Subgroups of a free group are free.
\end{thm}


\begin{dfn}
 Let $S$ be a subset of a group $G$ such that $G = \langle S \rangle$.
 \begin{enumerate}
  \item A \textit{presentation} for $G$ is a pair $(S, R)$, where $R$ is a set of words in $F(S)$ such that the normal closure of $\langle R \rangle$ in $F(S)$ (the smallest normal subgroup containing $\langle R \rangle$) equals the kernel of the homomorphism $\pi \colon F(S) \to G$ (where $\pi$ extends the identity map from $S$ to $S$). The elements of $S$ are called \textit{generators} and those of $R$ are called \textit{relations} of $G$.
  
  \item We say that $G$ is \textit{finitely generated} if there is a presentation $(S,R)$ such that $S$ is a finite set and we say $G$ is \textit{finitely presented} if there is a presentation $(S,R)$ with both $S$ and $R$ finite sets.
 \end{enumerate}
\end{dfn}



\end{document}
