\documentclass[../main]{subfiles}


\begin{document}

\section{Direct and Semidirect Products and \\ Abelian Groups}


\subsection{Direct Products}


\begin{dfn}
 ~\begin{enumerate}
   \item The \textit{direct product} $G_1 \times G_2 \times \cdots \times G_n$ of the groups $G_1, G_2, \ldots , G_n$ with operations $\star_1,\star_2, \ldots , \star_n$, respectively, is the set of $n$-tuples $(g_1,g_2, \ldots , g_n)$ where $g_i\in G_i$ with the operation defined componentwise: 
   \[(g_1,g_2, \ldots ,g_n)\star (h_1,h_2, \ldots ,h_n) = (g_1 \star_1 h_1, g_2 \star_2 h_2. \ldots g_n\star_n h_n).\]
   \item Similarly, the \textit{direct product} $G_1 \times G_2 \times \cdots$ of the groups $G_1, G_2, \ldots$ with operations $\star_1,\star_2, \ldots$, respectively, is the set of sequences $(g_1,g_2, \ldots)$ where $g_i\in G_i$ with the operation defined componentwise: 
   \[(g_1,g_2, \ldots)\star (h_1,h_2, \ldots) = (g_1 \star_1 h_1, g_2 \star_2 h_2. \ldots).\]
  \end{enumerate}
\end{dfn}


\begin{prop}
 If $G_1, \ldots, G_n$ are groups, their direct product is a group of order \\ $|G_1||G_2|\cdots |G_n|$ (if any $G_i$ is infinite, so is the direct product).
\end{prop}


\begin{prop}
 Let $G_1, G_2, \ldots , G_n$ be group and let $G = G_1 \times G_2 \times \cdots \times G_n$ be their direct product.
 \begin{enumerate}
  \item For each fixed $i$ the set of elements of $G$ which have the identity of $G_j$ in the $j^{\text{th}}$ position for all $j \neq i$ and arbitrary elements of $G_i$ in position $i$ is a subgroup of $G$ isomorphic $G_i$:
  \[G_i \cong \{(1,1,\ldots, 1, g_i,1,\ldots, 1) \mid g_i\in G_i\},\]
  (here $g_i$ appears in the $i^{\text{th}}$ position). If we identity $G_i$ with this subgroup, then $G_i \nor G$ and 
  \[G/G_i \cong G_1\times \cdots \times G_{i-1} \times G_{i+1} \times \cdots \times G_n.\]
  \item For each fixed $i$ define $\pi_i \colon G \to G_i$ by 
  \[\pi_i((g_1,g_2,\ldots,g_n)) = g_i.\]
  Then $\pi_i$ is a surjective homomorphism with
  \begin{align*}
   \text{ker}\pi_i &= \{(g_1,g_2, \ldots , g_{i-1}, 1, g_{i+1}) \mid g_j \in G_j \text{ for all } j\neq i\} \\
   &\cong G_1\times \cdots \times G_{i-1} \times G_{i+1} \times \cdots \times G_n
  \end{align*}
  (here 1 appears in position $i$).
  \item Under the identifications in part 1, if $x \in G_i$ and $y\in G_j$ for some $i \neq j$, then $xy = yx$. 
 \end{enumerate}
\end{prop}


\subsection{The Fundamental Theorem of Finitely Generated Abelian Groups}


\begin{dfn}
 ~\begin{enumerate}
   \item A group $G$ is \textit{finitely generated} if there is some finite subset $A$ of $G$ such that $G = \langle A \rangle$.
   \item For each $r\in \Z$ with $r \geq 0$ let $\Z^r = \Z \times \Z \times \cdots \times \Z$ be the direct product of r copies of the group $\Z$, where $\Z^0 = 1$. The group $\Z^r$ is called the \textit{free abelian group of order $r$}.
  \end{enumerate}
\end{dfn}


\begin{thm}
 (The Fundamental Theorem of Finitely Generated Abelian Groups) Let $G$ be a finitely generated abelian group. Then
 \begin{enumerate}
  \item \[G \cong \Z^r \times Z_{n_1} \times Z_{n_2} \times \cdots \times Z_{n_s}, \]
  for some $r,n_1,n_2, \ldots , n_s$ satisfying the following conditions:
    \begin{enumerate}
        \item $r \geq 0$ and $n_j \geq 2$ for all $j$, and 
        \item $n_{i+1} \mid n_i$ for all $1 \leq i \leq s-1$
    \end{enumerate}
    
    \item the expression in 1. is unique: if $G \cong \Z^t \times Z_{m_1} \times Z_{m_2} \times \cdots \times Z_{m_u}$, where $t$ and $m_1, m_2, \ldots , m_u$ satisfy (a) and (b), then $t = r$ and $m_i = n_i$ for all $i$. 
 \end{enumerate}
\end{thm}


\begin{dfn}
 The integer $r$ in Theorem 3 is called the \textit{free rank} or \textit{Betti number} of $G$ and the integers $n_1, n_2, \ldots , n_s$ are called the \textit{invariant factors} of $G$. The description of $G$ in Theorem 3(1) is called the \textit{invariant factor decomposition} of $G$.
\end{dfn}


\begin{nt}
 There is a bijection between the set of isomorphism classes of finite abelian groups of order $n$ and the set of integer sequences $n_1, n_2, \ldots , n_s$ such that 
 \begin{enumerate}
  \item $n_j \geq 2$ for all $j \in \{1, 2, \ldots , s\}$,
  \item $n_{i+1} \mid n_i, 1 \leq i \leq s-1$, and 
  \item $n_1 n_2 \cdots n_s = n$.
 \end{enumerate}
 Also notice that every prime divisor of $n$ must be a divisor of $n_1$ due to (2).
\end{nt}


\begin{cor}
 If $n$ is the product of distinct primes, then up to isomorphism the only abelian group of order $n$ is the cyclic group of order $n$, $Z_n$.
\end{cor}


\begin{thm}
 Let $G$ be an abelian group of order $n > 1$ and let the unique factorization of $n$ into distinct prime powers be 
 \[ n = p_1^{\alpha_1} p_2^{\alpha_2} \cdots p_k^{\alpha_k}. \]
 Then 
 \begin{enumerate}
  \item $G \cong A_1 \times A_2 \times \cdots \times A_k$, where $|A_i| = p_i^{\alpha_i}$
  
  \item for each $A \in \{ A_1, A_2, \ldots , A_k \}$ with $|A| = p^\alpha$,
  \[ A \cong Z_{p^{\beta_1}} \times Z_{p^{\beta_2}} \times \cdots \times Z_{p^{\beta_t}} \]
  with $\beta_1 \geq \beta_2 \geq \ldots \geq \beta_t \geq 1$ and $\beta_1 + \beta_2 + \ldots + \beta_t = \alpha$ (where $t$ and $\beta_1, \beta_2, \ldots , \beta_t$ depend on $i$)
  
  \item the decomposition in 1. and 2. is unique, i.e., if $G \cong B_1 \times B_2 \times \cdots \times B_m$, with $|B_i| = p_i^{\alpha_i}$ for all $i$, then $B_i \cong A_i$ and $B_i$ and $A_i$ have the same invariant factors.
 \end{enumerate}
\end{thm}


\begin{dfn}
 The integers $p^{\beta_j}$ described in the proceeding theorem are called the \textit{elementary divisors} of $G$. The description of $G$ in Theorem 5(1) and 5(2) is called the \textit{elementary divisor decomposition} of $G$. 
\end{dfn}


\begin{nt}
 For a group of order $p^\beta$ the invariant factors will be $p^{\beta_1}, p^{\beta_2}, \ldots , p^{\beta_t}$ such that
 \begin{enumerate}
  \item $\beta_j \geq 1$ for all $j \in \{1,2, \ldots ,t\}$,
  
  \item $\beta_i \geq \beta_{i+1}$ for all $i$, and 
  
  \item $\beta_1 + \beta_2 + \ldots + \beta_t = \beta$
 \end{enumerate}
\end{nt}


\begin{prop}
 Let $m,n \in \Z^+$.
 \begin{enumerate}
  \item $Z_m \times Z_n \cong Z_{mn}$ if and only if $(m,n) = 1$.
  
  \item If $n = p_1^{\alpha_1} p_2^{\alpha_2} \cdots p_k^{\alpha_k}$ then $Z_n \cong Z_{p_1^{\alpha_1}} \times Z_{p_2^{\alpha_2}} \times \cdots \times Z_{p_k^{\alpha_k}}$.
 \end{enumerate}
\end{prop}







 
 
 
 
 
 
 
 
 
 
 
 
 
 
 
 
 
 
 
 
 
 
 
 
 
 
 
 
 
 
 
 
 
 
 
 
 
 
 
 
 
\end{document}
