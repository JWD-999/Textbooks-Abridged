\documentclass[../main]{subfiles}


\begin{document}

\section{Direct and Semidirect Products and \\ Abelian Groups}


\subsection{Direct Products}


\begin{dfn}
 ~\begin{enumerate}
   \item The \textit{direct product} $G_1 \times G_2 \times \cdots \times G_n$ of the groups $G_1, G_2, \ldots , G_n$ with operations $\star_1,\star_2, \ldots , \star_n$, respectively, is the set of $n$-tuples $(g_1,g_2, \ldots , g_n)$ where $g_i\in G_i$ with the operation defined componentwise: 
   \[(g_1,g_2, \ldots ,g_n)\star (h_1,h_2, \ldots ,h_n) = (g_1 \star_1 h_1, g_2 \star_2 h_2. \ldots g_n\star_n h_n).\]
   \item Similarly, the \textit{direct product} $G_1 \times G_2 \times \cdots$ of the groups $G_1, G_2, \ldots$ with operations $\star_1,\star_2, \ldots$, respectively, is the set of sequences $(g_1,g_2, \ldots)$ where $g_i\in G_i$ with the operation defined componentwise: 
   \[(g_1,g_2, \ldots)\star (h_1,h_2, \ldots) = (g_1 \star_1 h_1, g_2 \star_2 h_2. \ldots).\]
  \end{enumerate}
\end{dfn}


\begin{prop}
 If $G_1, \ldots, G_n$ are groups, their direct product is a group of order \\ $|G_1||G_2|\cdots |G_n|$ (if any $G_i$ is infinite, so is the direct product).
\end{prop}


\begin{prop}
 Let $G_1, G_2, \ldots , G_n$ be group and let $G = G_1 \times G_2 \times \cdots \times G_n$ be their direct product.
 \begin{enumerate}
  \item For each fixed $i$ the set of elements of $G$ which have the identity of $G_j$ in the $j^{\text{th}}$ position for all $j \neq i$ and arbitrary elements of $G_i$ in position $i$ is a subgroup of $G$ isomorphic $G_i$:
  \[G_i \cong \{(1,1,\ldots, 1, g_i,1,\ldots, 1) \mid g_i\in G_i\},\]
  (here $g_i$ appears in the $i^{\text{th}}$ position). If we identity $G_i$ with this subgroup, then $G_i \nor G$ and 
  \[G/G_i \cong G_1\times \cdots \times G_{i-1} \times G_{i+1} \times \cdots \times G_n.\]
  \item For each fixed $i$ define $\pi_i \colon G \to G_i$ by 
  \[\pi_i((g_1,g_2,\ldots,g_n)) = g_i.\]
  Then $\pi_i$ is a surjective homomorphism with
  \begin{align*}
   \text{ker}\pi_i &= \{(g_1,g_2, \ldots , g_{i-1}, 1, g_{i+1}) \mid g_j \in G_j \text{ for all } j\neq i\} \\
   &\cong G_1\times \cdots \times G_{i-1} \times G_{i+1} \times \cdots \times G_n
  \end{align*}
  (here 1 appears in position $i$).
  \item Under the identifications in part 1, if $x \in G_i$ and $y\in G_j$ for some $i \neq j$, then $xy = yx$. 
 \end{enumerate}
\end{prop}


\subsection{The Fundamental Theorem of Finitely Generated Abelian Groups}


\begin{dfn}
 ~\begin{enumerate}
   \item A group $G$ is \textit{finitely generated} if there is some finite subset $A$ of $G$ such that $G = \langle A \rangle$.
   \item For each $r\in \Z$ with $r \geq 0$ let $\Z^r = \Z \times \Z \times \cdots \times \Z$ be the direct product of r copies of the group $\Z$, where $\Z^0 = 1$. The group $\Z^r$ is called the \textit{free abelian group of order $r$}.
  \end{enumerate}
\end{dfn}


\begin{thm}
 (The Fundamental Theorem of Finitely Generated Abelian Groups) Let $G$ be a finitely generated abelian group. Then
 \begin{enumerate}
  \item \[G \cong \Z^r \times Z_{n_1} \times Z_{n_2} \times \cdots \times Z_{n_s}, \]
  for some $r,n_1,n_2, \ldots , n_s$ satisfying the following conditions:
    \begin{enumerate}
        \item $r \geq 0$ and $n_j \geq 2$ for all $j$, and 
        \item $n_{i+1} \mid n_i$ for all $1 \leq i \leq s-1$
    \end{enumerate}
    
    \item the expression in 1. is unique: if $G \cong \Z^t \times Z_{m_1} \times Z_{m_2} \times \cdots \times Z_{m_u}$, where $t$ and $m_1, m_2, \ldots , m_u$ satisfy (a) and (b), then $t = r$ and $m_i = n_i$ for all $i$. 
 \end{enumerate}
\end{thm}


\begin{dfn}
 The integer $r$ in Theorem 3 is called the \textit{free rank} or \textit{Betti number} of $G$ and the integers $n_1, n_2, \ldots , n_s$ are called the \textit{invariant factors} of $G$. The description of $G$ in Theorem 3(1) is called the \textit{invariant factor decomposition} of $G$.
\end{dfn}


\begin{nt}
 There is a bijection between the set of isomorphism classes of finite abelian groups of order $n$ and the set of integer sequences $n_1, n_2, \ldots , n_s$ such that 
 \begin{enumerate}
  \item $n_j \geq 2$ for all $j \in \{1, 2, \ldots , s\}$,
  \item $n_{i+1} \mid n_i, 1 \leq i \leq s-1$, and 
  \item $n_1 n_2 \cdots n_s = n$.
 \end{enumerate}
 Also notice that every prime divisor of $n$ must be a divisor of $n_1$ due to (2).
\end{nt}


\begin{cor}
 If $n$ is the product of distinct primes, then up to isomorphism the only abelian group of order $n$ is the cyclic group of order $n$, $Z_n$.
\end{cor}


\begin{thm}
 Let $G$ be an abelian group of order $n > 1$ and let the unique factorization of $n$ into distinct prime powers be 
 \[ n = p_1^{\alpha_1} p_2^{\alpha_2} \cdots p_k^{\alpha_k}. \]
 Then 
 \begin{enumerate}
  \item $G \cong A_1 \times A_2 \times \cdots \times A_k$, where $|A_i| = p_i^{\alpha_i}$
  
  \item for each $A \in \{ A_1, A_2, \ldots , A_k \}$ with $|A| = p^\alpha$,
  \[ A \cong Z_{p^{\beta_1}} \times Z_{p^{\beta_2}} \times \cdots \times Z_{p^{\beta_t}} \]
  with $\beta_1 \geq \beta_2 \geq \ldots \geq \beta_t \geq 1$ and $\beta_1 + \beta_2 + \ldots + \beta_t = \alpha$ (where $t$ and $\beta_1, \beta_2, \ldots , \beta_t$ depend on $i$)
  
  \item the decomposition in 1. and 2. is unique, i.e., if $G \cong B_1 \times B_2 \times \cdots \times B_m$, with $|B_i| = p_i^{\alpha_i}$ for all $i$, then $B_i \cong A_i$ and $B_i$ and $A_i$ have the same invariant factors.
 \end{enumerate}
\end{thm}


\begin{dfn}
 The integers $p^{\beta_j}$ described in the proceeding theorem are called the \textit{elementary divisors} of $G$. The description of $G$ in Theorem 5(1) and 5(2) is called the \textit{elementary divisor decomposition} of $G$. 
\end{dfn}


\begin{nt}
 For a group of order $p^\beta$ the invariant factors will be $p^{\beta_1}, p^{\beta_2}, \ldots , p^{\beta_t}$ such that
 \begin{enumerate}
  \item $\beta_j \geq 1$ for all $j \in \{1,2, \ldots ,t\}$,
  
  \item $\beta_i \geq \beta_{i+1}$ for all $i$, and 
  
  \item $\beta_1 + \beta_2 + \ldots + \beta_t = \beta$
 \end{enumerate}
\end{nt}


\begin{prop}
 Let $m,n \in \Z^+$.
 \begin{enumerate}
  \item $Z_m \times Z_n \cong Z_{mn}$ if and only if $(m,n) = 1$.
  
  \item If $n = p_1^{\alpha_1} p_2^{\alpha_2} \cdots p_k^{\alpha_k}$ then $Z_n \cong Z_{p_1^{\alpha_1}} \times Z_{p_2^{\alpha_2}} \times \cdots \times Z_{p_k^{\alpha_k}}$.
 \end{enumerate}
\end{prop}


\subsection{Table of Groups of Small Order}


\begin{center}
\begin{tabular}{|c|C{4cm}|C{4cm}|C{4cm}|}\hline
Order & No. of Isomorphism Types & Abelian Groups & Non-abelian Groups\\\hline
1 & 1 & $Z_1$ & none\\\hline
2 & 1 & $Z_2$ & none\\\hline
3 & 1 & $Z_3$ & none\\\hline
4 & 2 & $Z_4$, $Z_2 \times Z_2$ & none\\\hline
5 & 1 & $Z_5$ & none\\\hline
6 & 2 & $Z_6$ & $S_3$\\\hline
7 & 1 & $Z_7$ & none\\\hline
8 & 5 & $Z_8$, $Z_4 \times Z_2$, $Z_2 \times Z_2 \times Z_2$ & $D_8$, $Q_8$\\\hline
9 & 2 & $Z_9$, $Z_3 \times Z_3$ & none\\\hline
10 & 2 & $Z_{10}$ & $D_{10}$\\\hline
11 & 1 & $Z_{11}$ & none\\\hline
12 & 5 & $Z_{12}$, $Z_6 \times Z_2$ & $A_4$, $D_{12}$, $Z_3 \rtimes Z_4$\\\hline
13 & 1 & $Z_{13}$ & none\\\hline
14 & 2 & $Z_{14}$ & $D_{14}$\\\hline
15 & 1 & $Z_{15}$ & none\\\hline
16 & 14 & $Z_{16}$, $Z_8 \times Z_2$, $Z_4 \times Z_4$, $Z_4 \times Z_2 \times Z_2$, $Z_2 \times Z_2 \times Z_2 \times Z_2$ & not listed\\\hline
17 & 1 & $Z_{17}$ & none\\\hline
18 & 5 & $Z_{18}$, $Z_6 \times Z_3$ & $D_{18}$, $S_3 \times Z_3$, $(Z_3 \times Z_3) \rtimes Z_2$\\\hline
19 & 1 & $Z_{19}$ & none\\\hline
20 & 5 & $Z_{20}$, $Z_{10} \times Z_2$ & $D_{20}$, $Z_5 \rtimes Z_4$, $F_{20}$\\\hline
\end{tabular}
\end{center}


\begin{nt}
 The group $F_{20}$ of order 20 has generators and relations 
 \[ \langle x,y \mid x^4 = y^5 = 1, xyx^{-1} = y^2 \rangle. \]
 This group is called the \textit{Frobenius group} of order 20 and can be viewed as the subgroup $F_{20} = \langle (2354), (12345) \rangle$ of $S_5$.
\end{nt}


\subsection{Recognizing Direct Products}


\begin{dfn}
 Let $G$ be a group, let $x,y \in G$ and let $A, B$ be nonempty subsets of $G$. 
 \begin{enumerate}
  \item Define $[x,y] = x^{-1}y^{-1}xy$, called the \textit{commutator} of $x$ and $y$.
  \item Define $[A,B] = \langle [a,b] \mid a \in A, b\in B \rangle$, the group generated by commutators of elements of $A$ and from $B$.
  \item Define $G' = \langle [x,y] \mid x,y \in G \rangle$, the subgroup of $G$ generated by commutators of elements from $G$, called the \textit{commutator subgroup} of $G$.
 \end{enumerate}
\end{dfn}


\begin{prop}
 Let $G$ be a group, let $x,y \in G$ and let $H \leq G$. Then 
 \begin{enumerate}
  \item $xy = yx[x,y]$ (in particular, $xy = yx$ if and only if $[x,y] =1 $).
  
  \item $H \nor G$ if and only if $[H,G] \leq H$.
  
  \item $\sigma [x,y] = [ \sigma(x), \sigma(y)]$ for any automorphism $\sigma$ of $G$, $G'$char$G$ and $G/G'$ is abelian
  
  \item  $G/G'$ is the largest abelian quotient of $G$ in the sense that if $H \nor G$ and $G/H$ is abelian, then $G' \leq H$. Conversely, if $G' \leq H$, then $H \nor G$ and $G/H$ is abelian.
  
  \item If $\phi \colon G \to A$ is any homomorphism of $G$ into an abelian group $A$, then $\phi$ factors through $G'$ i.e., $G' \leq \text{ker}\phi$ and the following diagram commutes:
  
\[\begin{tikzcd}
	G && {G/G'} \\
	\\
	&& A
	\arrow[from=1-1, to=1-3]
	\arrow["\phi"', from=1-1, to=3-3]
	\arrow[from=1-3, to=3-3]
\end{tikzcd}\]
 \end{enumerate}
\end{prop}


\begin{prop}
 Let $H$ and $K$ be subgroups of the group $G$. The number of distinct ways of writing each element of the set $HK$ in the form $hk$, for some $h \in H$ and $k \in K$ is $|H \cap K|$. In particular, if $H \cap K = 1$, then each element of $HK$ can be written uniquely as the product $hk$, for some $h \in H$ and $k \in K$. 
\end{prop}


\begin{thm}
 Suppose $G$ is a group with subgroups $H$ and $K$ such that 
 \begin{enumerate}
  \item $H$ and $K$ are normal in $G$, and 
  \item $H \cap K = 1$.
 \end{enumerate}
 Then $HK \cong H \times K$.
\end{thm}

\begin{nt}
 The above conditions are simply the necessary conditions to ensure that the map 
 \begin{align*}
  \phi \colon & HK \to H \times K \\
  & hk \mapsto (h,k)
 \end{align*}
 is well defined and an isomorphism.
\end{nt}


\begin{dfn}
 If $G$ is a group and $H$ and $K$ are normal subgroups of $G$ with $H \cap K = 1$, we call $HK$ the \textit{internal direct product} of $H$ and $K$. We shall (when emphasis is called for) call $H \times K$ the \textit{external direct product} pf $H$ and $K$. (The distinction here is purely notational by Theorem 9).
\end{dfn}


\subsection{Semidirect Products}


\begin{thm}
 Let $H$ and $K$ be groups and let $\phi$ be a homomorphism from $K$ into Aut$(H)$. Let $\cdot$ denote the (left) action of $K$ on $H$ determined by $\phi$. Let $G$ be the set of order pairs $(h,k)$ with $h \in H$ and $k \in K$ and define the following multiplication on $G$:
 \[ (h_1, k_1) (h_2, k_2) = (h_1 k_1 \cdot h_2, k_1 k_2).\]
 \begin{enumerate}
  \item This multiplication makes $G$ into a group of order $|G| = |H||K|$.
  
  \item The sets $\{(h,1) \mid h \in H \}$ and $\{(1,k) \mid k \in K\}$ are subgroups of $G$ and the maps $h \mapsto (h,1)$ for $h \in H$ and $k \mapsto (1,k)$ for $k \in K$ are isomorphisms of these subgroups with the groups $H$ and $K$ respectively;
  \[ H \cong  \{(h,1) \mid h \in H \} \quad \text{and} \quad K \cong \{(1,k) \mid k \in K\}. \]
 \end{enumerate}
 
 Identifying $H$ and $K$ with their isomorphic copies in $G$ described in 2. we have 
 
  \begin{enumerate}
  \setcounter{enumi}{2} %adds two to the counter
   \item $H \nor G$
   
   \item $H \cap K = 1$
   
   \item for all $h \in H$ and $k \in K$, $khk^{-1} = k \cdot h = \phi(k)(h)$
  \end{enumerate}
\end{thm}


\begin{dfn}
 Let $H$ and $K$ be groups and let $\phi$ be a homomorphism from $K$ into Aut$(H)$. The group described in Theorem 10 is called the \textit{semidirect product} of $H$ and $K$ with respect to $\phi$ and will be denoted by $H \rtimes_\phi K$ (when there is no danger of confusion we shall simply write $H \rtimes K$).
\end{dfn}


\begin{prop}
 Let $H$ and $K$ be groups and let $\phi \colon K \to$Aut$(H)$ be a homomorphism. Then the following are equivalent:
 \begin{enumerate}
  \item the identity (set) map between $H \rtimes K$ and $H \times K$ is a group homomorphism (hence and isomorphism)
  
  \item $\phi$ is the trivial homomorphism from $K$ into Aut$(H)$
  
  \item $K \nor H \rtimes k$.
 \end{enumerate}
\end{prop}


\begin{thm}
 Suppose $G$ is a group with subgroups $H$ and $K$ such that
 \begin{enumerate}
  \item $H \nor G$, and 
  
  \item $H \cap K = 1$.
 \end{enumerate}
 Let $\phi \colon K \to$Aut$(H)$ be the homomorphism defined by mapping $k \in K$ to the automorphism of left conjugation by $k$ on $H$. Then $HK \cong H \rtimes K$. In particular, if $G = HK$ with $H$ and $K$ satisfying 1. and 2., then $G$ is the semidirect product of $H$ and $K$. 
\end{thm}


\begin{dfn}
 Let $H$ be a subgroup of the group $G$. A subgroup $K$ of $G$ is called a \textit{complement} for $H$ in $G$ if $G = HK$ and $H \cap K = 1$.
\end{dfn}

\begin{nt}
 With the above terminology, the criterion for recognizing a semidirect product is simply that there must exist a complement for some proper normal subgroup of $G$. 
\end{nt}

\end{document}
