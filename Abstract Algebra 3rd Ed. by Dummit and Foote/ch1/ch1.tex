\documentclass[../main]{subfiles}

\begin{document}
 
 \section{Group Theory}
 
 \subsection{Basic Axioms and Examples}
 
 
 \begin{dfn}
  \begin{enumerate}
   \item A \textit{binary operation} $\star$ on a set $G$ is a function $\star \colon G\times G \to G$. For any $a,b\in G$ we shall write $a\star b$ for $\star (a,b)$.
   \item A binary operation $\star$ on a set $G$ is associative if for all $a,b,c\in G$ we have $a\star (b\star c)=(a\star b)\star c$.
   \item If $\star$ is a binary operation on a set $G$ we say elements $a$ and $b$ of $G$ \tectit{commute} if $a\star b = b\star a$. We say $\star$ (or $G$) is \textit{commutative} if for all $a,b\in G$, $a\star b = b\star a$.
  \end{enumerate}
 \end{dfn}
 
 \begin{prop}
  If G is a group under the operation $\cdot$, then
  \begin{enumerate}
   \item The identity of G is unique 
   \item for each $a \in G,$ $a^{-1}$ is uninuely determined
   \item $(a^{-1})^{-1} = a$ for all $a \in G$
   \item $(a\cdot b)^{-1}=(b^{-1})\cdot(a^{-1})$
   \item for any $a_q,a_2, \ldots,a_n\in G$ the value of $a_1 a_2 \cdots a_n$ is independent of how the expresion is bracketed 
  \end{enumerate}
 \end{prop}

 \begin{prop}
  Let G be a group and let $a,b\in G$. The equations $ax=b$ and $ya=b$ have unique solutions for $x,y \in G$. In particular, the left and right cancelation laws hold in G, i.e.,
  \begin{enumerate}
   \item if $au=av$, then $u=v$, and 
   \item if $ub=vb$, then $u=v$.
  \end{enumerate}
 \end{prop}

 \begin{dfn}
  For $G$ a group and $x\in G$ define the \textit{order} of $x$ to be the smallest positive integer $n$ such that $x^n = 1$, denoted $|x|$. If there is no such integer than we define the order of x to be infinity. 
 \end{dfn}
 
 
 \subsection{Homomorphism and Isomorphisms}
 
 \begin{dfn}
  
 \end{dfn}

 
 
 
\end{document}
