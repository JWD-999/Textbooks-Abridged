\documentclass[../main]{subfiles}

\begin{document}
 
 \section{Group Theory}
 
 \subsection{Basic Axioms and Examples}
 
 
 \begin{dfn}
  \begin{enumerate}
   \item A \textit{binary operation} $\star$ on a set $G$ is a function $\star \colon G\times G \to G$. For any $a,b\in G$ we shall write $a\star b$ for $\star (a,b)$.
   \item A binary operation $\star$ on a set $G$ is associative if for all $a,b,c\in G$ we have $a\star (b\star c)=(a\star b)\star c$.
   \item If $\star$ is a binary operation on a set $G$ we say elements $a$ and $b$ of $G$ \textit{commute} if $a\star b = b\star a$. We say $\star$ (or $G$) is \textit{commutative} if for all $a,b\in G$, $a\star b = b\star a$.
  \end{enumerate}
 \end{dfn}
 
 \begin{prop}
  If G is a group under the operation $\cdot$, then
  \begin{enumerate}
   \item The identity of G is unique 
   \item for each $a \in G,$ $a^{-1}$ is uninuely determined
   \item $(a^{-1})^{-1} = a$ for all $a \in G$
   \item $(a\cdot b)^{-1}=(b^{-1})\cdot(a^{-1})$
   \item for any $a_q,a_2, \ldots,a_n\in G$ the value of $a_1 a_2 \cdots a_n$ is independent of how the expresion is bracketed 
  \end{enumerate}
 \end{prop}

 \begin{prop}
  Let G be a group and let $a,b\in G$. The equations $ax=b$ and $ya=b$ have unique solutions for $x,y \in G$. In particular, the left and right cancelation laws hold in G, i.e.,
  \begin{enumerate}
   \item if $au=av$, then $u=v$, and 
   \item if $ub=vb$, then $u=v$.
  \end{enumerate}
 \end{prop}

 \begin{dfn}
  For $G$ a group and $x\in G$ define the \textit{order} of $x$ to be the smallest positive integer $n$ such that $x^n = 1$, denoted $|x|$. If there is no such integer than we define the order of x to be infinity. 
 \end{dfn}
 
 
 \setcounter{subsection}{5}
 
 
 \subsection{Homomorphism and Isomorphisms}
 
 
 \begin{dfn}
  Let $(G,\star)$ and $(H,\diamond)$ be groups. A map $\phi \colon G \to H$ such that $\phi (x\star y)= \phi(x)\diamond \phi(y)$, for all $x,y\in G$ is called a \textit{homomorphism}. Moreover, if $\phi$ is bijective it is called an \textit{isomorphism} and we say that $G$ and $H$ are \textit{isomorphic} or of the same \textit{isomorphism type}, written $G \cong H$.
 \end{dfn}


 \begin{rmk}
  If $\phi \colon G\to H$ is an isomorphism then 
  \begin{enumerate}
   \item $|G| = |H|$
   \item $G$ is abelian if and only if $H$ is abelian
   \item for all $x\in G, |x| = |\phi (x)|$
  \end{enumerate}

 \end{rmk}


 \subsection{Group Actions}
 
 
 \begin{dfn}
  A \textit{group action} of a group $G$ on a set $A$ is a map from $G\times A$ to $A$ (written as $g\cdot a$, for all $g\in G$ and $a\in A$) satisfying the following properties:
  \begin{enumerate}
   \item $g_1\cdot(g_2\cdot a)=(g_1 g_2)\cdot a$, for all $g_1, g_2\in G, a\in A$, and 
   \item $1\cdot a =a$ for all $a\in A$.
  \end{enumerate}
 \end{dfn}

 
 \begin{rmk}
  Let the group $G$ act on the set $A$. From each fixed $g\in G$ we get a map $\sigma_g$ defined by 
  \begin{align*}
   &\sigma_g \colon A \to A \\
   &\sigma_g(a)=g\cdot a.
  \end{align*}
  The following are true
  \begin{enumerate}
   \item for each fixed $g\in G$, $\sigma_g$ is a permutation of A, and 
   \item the map from $G$ to $S_A$ defined bt $g\mapsto \sigma_g$ is a homomorphism. Moreover this map is called the \textit{permuation representation} associated to the given action. 
  \end{enumerate}
 \end{rmk}

 \begin{rmk}
  As a consequence of the above remark, if $\phi \colon G \to S_A$ is a homomorphism (here $S_A$ is the symmetric group on the set A), then the map from $G\times A$ to $ A$ defined by 
  \begin{align*}
   g\cdot a = \phi(g)(a) \text{for all} g\in G \text{, and all} a\in A
  \end{align*}
  is a group action of $G$ on $A$.
 \end{rmk}

 
\end{document}
