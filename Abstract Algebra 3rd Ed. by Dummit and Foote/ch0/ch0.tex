\documentclass[../main]{subfiles}

\begin{document}
 
 \section{Preliminaries}
 
 \subsection{Basics}
 
 \begin{prop}
  Let $f \colon A \to B.$
  \begin{enumerate}
   \item The map $f$ is injective if and only if $f$ has a left inverse.
   \item The map $f$ is surjective if and onbly if $f$ has a right inverse.
   \item The map $f$ is a bijection if and only if there exist $g \colon B \to A$ such that $f\circ g$ is the indentity map on B and $g\circ f$ is the identity map on A.
   \item If A and B are finte sets with the same number of elements the $f\colon A \to B$ is bijective if and only if $f$ is injective if and only if $f$ is surjective.
  \end{enumerate}
 \end{prop}

 \begin{prop}
  Let A be a nonempty set.
  \begin{enumerate}
   \item If $\sim$ defines an equivalence relation on A then the set of equivalence classes of $\sim$ form a partision of A.
   \item If $\{A_i \mid i \in I\}$ is a parttion of A then there is an equivalence relation on A whose equivalence classes are precisely the sets $A_i, i\in I$
  \end{enumerate}
 \end{prop}
 
\end{document}
