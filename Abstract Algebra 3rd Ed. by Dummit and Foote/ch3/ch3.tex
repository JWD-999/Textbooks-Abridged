\documentclass[../main]{subfiles}

\begin{document}
 
 \section{Quotient Groups and Homomorphisms}
 
 
 \subsection{Definitions and Examples}
 

 \begin{dfn}
  If $\phi$ is a homomorphism $\phi \colon G \to H$, the \textit{kernel} of $\phi$ is the set
  \begin{align*}
   \{g\in G \mid \phi(g)= 1\}
  \end{align*}
  and will be denoted by ker$\phi$ (here 1 is the identity of H).
 \end{dfn}

 
 \begin{prop}
  Let $G$ and $H$ be groups and let $\phi \colon H \to H$ be a homomorphism.
  \begin{enumerate}
   \item $\phi(1_G)=1_H$, where $1_G$ and $1_H$ are the identities of $G$ and $H$, respectively.
   \item $\phi(g^{-1})=\phi(g)^{-1}$ for all $g\in G$.
   \item $\phi(g^n)=\phi(g)^n$ for all $n\in \Z$.
   \item ker$\phi$ is a subgroup of $G$.
   \item im$\phi$, the image of $G$ uner $\phi$, is a subgorup of $H$.
  \end{enumerate}
 \end{prop}

 
 \begin{dfn}
  Let $\phi \colon G\to H$ be a homomorphism with kernel $K$. The \textit{quotient group} or \textit{factor group}, $G/K$ (read $G$ \textit{modulo} $K$ or simply $G$ \textit{mod} $K$), is the group whose elements are the fibers of $\phi$ with the following group operation: If $X$ is the fiber above $a$ and $Y$ is the fiber above $b$ then the product $XY$ in $G/K$ is defined to be the fiber above the product $ab$ in $G$.
 \end{dfn}

 
 \begin{prop}
  Let $\phi \colon G \to H$ be a homomorphism with kernel $K$. Let $X\in G/K$ be the fiber above $a$, i.e., $X=\phi^{-1}(a)$. Then 
  \begin{enumerate}
   \item For any $u\in X$,  $X=\{uk\mid k\in K\}$
   \item For any $u\in X$,  $X=\{ku\mid k\in K\}$
  \end{enumerate}
 \end{prop}
 
 
 \begin{dfn}
  For any $N\leq G$ and any $g\in G$ let 
  \begin{align*}
   gN=\{gn \mid n\in N\} \text{ and } Ng=\{ng \mid n\in N\}
   \end{align*}
  called respectively a \textit{left coset} and a \textit{right coset} of $N$ in $G$. Any element of a coset is called a \textit{representative} for the coset.
 \end{dfn}
 
 
 \begin{thm}
  Let $G$ be a group and let $K$ be the kernel of some homomorphism from $G$ to another group. Then the set of whose elements are ;eft coeset of $K$ in $G$ with operation defined by 
  \begin{align*}
   uK\circ vK = (uv)K
  \end{align*}
  forms a group, $G/K$. This operation is well defined and does not depend on the choice of representatives. 
 \end{thm}
 
 
 \begin{prop}
  Let $N$ be any subgroup of the group $G$. The set of left cosets of $N$ in $G$ form a partition of $G$. Furthermore, for all $u,v\in G, uN=vN$ if and only if $v^{-1}u\in N$ and in particular, $uN = vN$ if and only if $u$ and $v$ are representatives of the same coset.
  \end{prop}
  
  
  \begin{prop}
   Let $G$ be a group and let $N$ be a subgroup of $G$. 
   \begin{enumerate}
    \item The operation on the set of left cosets of $N$ in $G$ described by
    \begin{align*}
     uN\cdot vN =(uv)N
    \end{align*}
    is well defined if and only if $gng^{-1}$ for all $g\in G$ and all $n\in N $. 
    \item If the above operation is well defined, then it makes the set of left cosets of $N$ in $G$ into a group. In particular the identity of this group is the coset $1N$ and the inverse of $gN$ is the coset $g^{-1}$, i.e, $(gN)^{-1} = g^{-1}N$.
   \end{enumerate}
  \end{prop}
  
  
  
  \begin{dfn}
   The element $gng^{-1}$ is called the \textit{conjugate} of $n\in N$ by $g$. The set $gNg^{-1}=\{gng^{-1}\mid n\in N\}$ is called the \textit{conjugate} of $N$ by $g$. The element $g$ is said to \textit{normalize} $N$ if $gNg^{-1} = N$. A subgroup $N$ of a group $G$ is called \textit{normal} if every element of $G$ normalizes $N$, i.e., if $gNg^{-1} = N$ for all $g\in G$. If $N$ is a normal subgoup of $G$ we shall write $N \nor G$. 
  \end{dfn}
  
  
  \begin{thm}
   Let $N$ be a subgroup of the group $G$. The following are equivalent:
   \begin{enumerate}
    \item $N \nor G$
    \item $N_G(N)=G$ (recall $N_G(N)$ is the normalizer in $G$ of $N$)
    \item $gN=Ng$ for all $g\in G$
    \item the operation on left cosets of $N$ in $G$ described in Proposition 5 makes the set of left cosets into a group
    \item $gNg^{-1} \subseteq N$ for all $g\in G$.
   \end{enumerate}
  \end{thm}
  
  
  \begin{prop}
   A subgroup $N$ of the group $G$ is normal if and only if it is the kernel of some homomorphism.
  \end{prop}
  
  
  \begin{dfn}
   Let $N \nor G$. The homomorphism $\pi \colon G \to G/N$ defined by $\pi(g)=gN$ is called the \textit{natural projection (homomorphism)} of $G$ onto $G/N$. If $\overline{H} \leq G/N$, then \textit{complete preimage} of $\overline{H}$ in $G$ is the preimage of $\overline{H}$ under the natural projection homomorphism. 
  \end{dfn}
  
  
  \subsection{More on Cosets and Lagrange's Thoerem}
  
  
  \begin{thm}
   (\textit{Lagrange's Theorem}) If $G$ is a fintite group and $H$ is a subgroup of $G$, then the order of $H$ divides the order of $G$ and the number of left cosets of $H$ in $G$ equals $\frac{|G|}{|H|}$.
  \end{thm}
  
  
  \begin{dfn}
   If $G$ is a group and $H\leq G$, the number of left cosets of $H$ in $G$ is called the \textit{index} of $H$ in $G$ and is denoted by $|G:H|$.
  \end{dfn}
  
  
  \begin{cor}
   If $G$ is a finite group and $x\in G$, then the order of $x$ divides the order of $G$. In particular, $x^{|G|}=1$ for all $x$ in $G$.
  \end{cor}
  
  
  \begin{cor}
   If $G$ is a group of prime order  $p$, then $G$ is cyclic, hence $G \cong Z_p$ (note that this text uses $Z_n$ to denote the cyclic group of order $n$ written in multipicative notation and that given any $n\in \Z$, $Z_n \cong \Z/n\Z)$.
  \end{cor}
  
  
  \begin{nt}
   For finite abelian groups the full converse of Lagrange's theorem holds, that is the group has a subgroup of order $n$ for each $n$ that divides the order of the group. 
  \end{nt}
  
  
  \begin{thm}
   (Cauchy's Thoerem) If $G$ is a fintite group and $p$ is a prime dividing $|G|$, then $G$ has an element of order $p$.
  \end{thm}
  
  
  \begin{thm}
   (Sylow) If $G$ is a finite group of order $p^\alpha m$, where $p$ is a prime not dividing $m$, then $G$ has a subgroup of order $p^\alpha$.
  \end{thm}
  
  
  \begin{dfn}
   Let $H$ and $K$ be subgroups of a group and define
   \begin{align*}
    HK = \{hk \mid h\in H, k\in K\}.
   \end{align*}
  \end{dfn}
  
  
  \begin{prop}
   If $H$ and $K$ are finte subgroups of a group then 
   \begin{align*}
    |HK|=\frac{|H||K|}{|H\cap K|}.
   \end{align*}
  \end{prop}
  
  
  \begin{prop}
   If $H$ and $K$ are subgroups of a group, $HK$ is a subgroup if and only if $HK=KH$.
  \end{prop}
  
  
  \begin{nt}
   $HK=KH$ does not imply that the elements of $H$ commute with the elements of $K$
  \end{nt}
  
  
  \begin{cor}
   If $H$ and $K$ are subgroups of $G$ and $H\leq N_G(K)$, then $Hk$ is a subgroup of $G$. In particular, if $K \nor G$, Then $HK\leq G$ for any $H \leq G$ (Since if $K \nor G$, $N_G(k)=G$).
  \end{cor}
  
  
  \begin{dfn}
   If $A$ is any subset of $N_G(K)$ (or $C_G(K)$), we shall say $A$ \textit{normalizes} $K$ (\textit{centralizes} $K$, respectively).
  \end{dfn}
  
  
  \subsection{The Isomorphism Thoerems}
  
  
  \begin{thm}
   (The First Isomorphism Theorem) If $\phi \colon G\to H$ is a homomorphism, then ker$\phi \nor G$ and $G/$ker$\phi \cong \phi(G)$.
  \end{thm}
  
  
  \begin{cor}
   Let $\phi \colon G\to H$ be a homomorphism.
   \begin{enumerate}
    \item $\phi$ is injective if and only if ker$\phi = 1$.
    \item $|G:$ker$\phi = |\phi(G)|$.
   \end{enumerate}
  \end{cor}
  
  
  \begin{thm}
   
  \end{thm}





  
\end{document}
